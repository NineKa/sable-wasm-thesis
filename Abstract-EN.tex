\textit{WebAssembly} is a relatively new language, introduced to improve the performance of compute-intensive workloads in web-based applications.  It offers a compact binary bytecode intended to allow for fast compilation and improved optimization opportunities over dynamic web languages like JavaScript.  These properties, however, also make it an interesting target for static execution, enabling web code to run outside of a browser as well as within it.  In this thesis, we describe \textit{SableWasm}, a static, multi-pass compiler system that translates sandboxed WebAssembly applications to native shared libraries. Our work covers several different aspects of compiler design. First, we provide an efficient and extensible WebAssembly module parsing and validation framework, with improved execution speed and memory footprint compared to the reference baseline. We then define a middle-level intermediate representation and build an analysis and transformation framework. We explore several classic data-flow analyses, such as dominator-tree construction and local value numbering within the framework, and additionally identify several WebAssembly specific optimization opportunities, which we address through custom transformation passes, such as redundant local variable elimination. SableWasm also incorporates several in-progress extension proposals including the SIMD vector operation extension. Optimized intermediate code is then converted to native code through a backend implementation with the help of the LLVM compiler framework and a runtime that enables C/C++ programs to interact with the WebAssembly module directly. Finally, we evaluate SableWasm by benchmarking against several well-known testing suites and observe performance improvement compared to the baseline implementation.