\chapter{Introduction}
\label{chapter:introduction}

Web-based applications have grown in popularity in recent years. From the early
days of simple web applets to current full-blown programs, their codebase's
complexity and size has grown rapidly. Due to the design of most browsers,
programmers have to choose JavaScript or its dialects to implement them. This
approach is quite successful; however, it still leaves several problems
unsolved. First, JavaScript is a scripting language and employs many dynamic
features that prevent backend runtime environment from efficient execution,
such as dynamic typing. Additionally, when porting existing applications to
JavaScript, especially those with a large codebase where manually translating
source code line-by-line is not feasible, a nontrivial source-to-source
compiler is needed due to the structural difference between native binaries and
JavaScript source codes.

To address these problems, the WebAssembly working group was established in
2017, and purposed a new standard for distributing applications over the
Internet. WebAssembly focuses on safety, performance, portability and module
compactness. These properties also make it an interesting target for static
execution, enabling sandboxed applications outside of browsers. To this end,
the WebAssembly community further designed the WebAssembly System Interface
(WASI), which provides a standardized interface for WebAssembly modules to
access native features such as the file system.

WebAssembly is also an evolving language. Although the WebAssembly community
has published the minimum viable product (MVP) WebAssembly, the community is
still actively proposing and experimenting with new language features, such as
exception handling and garbage collection. These additional language features
are proposed in language extension proposals that modify the current WebAssembly
specification syntactically and semantically. Thus, a well-designed WebAssembly
runtime environment system should be modular and extensible, leaving space for
future design changes.

\section{Contribution}

\begin{figure}
    \centering
    \includegraphics[width=\textwidth]{Images/design}
    \caption{The SableWasm compiler and runtime}
    \label{fig:design}
\end{figure}

This thesis aims to design and implement a runtime environment that enables
WebAssembly to run outside of the browser. To this end, this thesis makes three
major contributions. Figure~\ref{fig:design} illustrates the SableWasm compiler
and runtime system. We mark our contributions in this thesis as shaded boxes
in the figure.

\paragraph{Implementing a WebAssembly runtime system}
Our first contribution is a standalone WebAssembly runtime environment with
support for the WebAssembly System Interface (WASI). We first start by
implementing a custom extensible parser frontend for WebAssembly binary format,
shown as the `Parser Frontend' in figure~\ref{fig:design}.
We then define a `middle-level' representation (MIR) for SableWasm.
To match modern hardware, SableWasm MIR is a register-based
\emph{control flow graph} representation of the program, while, on the
other hand, WebAssembly operates over a stack-based virtual machine. Hence,
translating between them is nontrivial. Therefore, we design and implement a
frontend code generator that lowers WebAssembly bytecode into SableWasm MIR,
shown as the `Frontend Code Generator' in figure~\ref{fig:design}.
SableWasm MIR plays a critical role in the SableWasm system. First, it
provides a middle ground where we implement an extensible and straightforward
optimization framework. With the help of the framework, we experiment with
several analyses and optimizations on SableWasm MIR. Second,
SableWasm MIR also separates the frontend from the backend. Currently, we
implement an ahead-of-time (AOT) compiler backend using the LLVM compiler
infrastructure \cite{llvm-thesis}, shown as the `Backend Code Generator' in
figure~\ref{fig:design}. However, there are several challenges when lowering
SableWasm MIR into LLVM intermediate representation. For example, SableWasm MIR,
similar to WebAssembly bytecode, utilizes several abstract high-level concepts
such as linear memory and indirect function calls. These operations cannot be
trivially mapped to LLVM instructions and require runtime library support.
Hence, the last component of SableWasm is a runtime library that provides
builtin runtime functions for the generated modules and defines an easy-to-use
interface for the host system, shown as the `SableWasm Runtime' in
figure~\ref{fig:design}.

\paragraph{Adding support for WebAssembly extensions}
Our second contribution in this thesis is to experiment and adopt several
in-progress WebAssembly language extensions. SableWasm is designed to be
extensible and currently implements four post-MVP WebAssembly features. The
most interesting one among them is perhaps the fixed-width SIMD operation
extension which defines vector-based operations that can operate on multiple
data simultaneously, packed into special vector registers and supported by
modern hardware. The SIMD extension in WebAssembly
introduces one additional value type and approximately 240
new instructions to the specification. As we have discussed earlier in this
section, SableWasm MIR provides a middle ground where we perform optimization
on the program. Therefore, we would like to keep the size of the SableWasm MIR
instruction set simple. To achieve this goal, we carefully design a set of
reduction patterns in the frontend code generator that significantly reduce the
number of instructions needed. We also generalize our backend code generator
that targets LLVM by emitting corresponding vector operation instructions.

\paragraph{Evaluating system performance}
Our last contribution in this thesis is to investigate how SableWasm performs
and the factors that affect the performance. Here we focus on three research
questions: First, how does SableWasm perform comparing to other existing
WebAssembly runtime implementations? Second, does optimization over the input
WebAssembly modules affect SableWasm's overall performance? Finally, does the
SIMD operation extension bring performance improvement to the system? To answer
these questions, we analyze the performance of three well-known benchmark
suites, Polybench \cite{polybench}, Ostrich \cite{ostrich}, and NPB \cite{npb}.
We also examine generated LLVM intermediate representations in SableWasm to
search for factors contributing to the slow down in the system.

\section{Thesis outline}

This thesis consists of nine chapters in total, including the introduction
chapter.
Chapter 2 discusses the background information that helps the
understanding rest of the thesis. It first presents the motivation for
WebAssembly and WebAssembly System Interface (WASI), followed by a brief
overview of the LLVM intermediate representation.
Chapter 3 to chapter 6 discusses the design of implementation of the
SableWasm system. Chapter 3 starts with presenting the custom extensible and
efficient parser frontend for WebAssembly binary format.
Chapter 4 continues the discussion of SableWasm by describing SableWasm MIR's
design.
Chapter 5 discusses the code generating strategies used when
lowering WebAssembly bytecode to SableWasm MIR and the optimization framework
Chapter 5 also presents several optimization passes we experimented with the
framework, such as control flow graph simplification and type inference.
Chapter 6 illustrates the last component of SableWasm, the LLVM backend and the
runtime support library.
In chapter 7, we investigate the performance of SableWasm by presenting
benchmark results and discussing several possible theories for the slowdown.
Finally, chapter 8 discusses related work and chapter 9 presents our conclusion
along with future work.