\subsection{Dominators and Dependence}

Dominator tree and immediate dominance are close related to \emph{static single
    assignment} (SSA) form, and Ron Cytron's classic paper on converting
control flow graph (CFG) to SSA \cite{ibm-ssa} shows that SSA directly derives
from them. The dominator tree represents the dominance relationship between
basic blocks. A basic block is a \emph{dominator} of another if all control flow
reaching the later block must go through the first block. On the other hand,
\emph{immediate dominance} defines a stricter relationship between basic blocks.
A basic block is an immediate dominator of another if it satisfies two
conditions. First, the candidate block must be a dominator block of the second
one. Second, it does not dominate any other blocks that dominate the second
block. Although the SableWasm MIR is already in SSA form, the dominator tree is
still helpful in the later analysis and backend code generation. One may notice
that the dominator relationship in SSA is comparable to the same problem in
graph theory. Indeed, they are the same problem if we treat the basic blocks as
vertices and control flow paths as edges among them.  A direct solution to
compute the dominator set utilizes forward analysis within $O(n^2)$, respecting
to the number of basic blocks in the CFG. More efficient algorithms can yield
the dominator set within almost linear time, such as Tarjan's algorithm
\cite{tarjan-fast-dominator}, and its refined version
\cite{tarjan-fast-dominator-improved}. Currently, SableWasm compiles programs
usually with smaller functions that contain approximately 200 basic blocks at
most. Hence, an efficient complex algorithm does not have too much room for
improvement. In the future, if this becomes the bottleneck of the compilation
pipeline, one should replace the implementation with a better algorithm. This
section will present the forward analysis implementation briefly, and it is a
classic implementation for dominator tree construction.

\paragraph{Formalisms}
In the rest of the section, we will use $dom(\cdot)$ to represent the set of
strict dominators for a given basic block. The set of strict dominators for
$A$ is the set of dominators for $A$ subtracting $A$ itself. Hence, block $A$
is a strict dominator for block $B$ implies that $A \in dom(B)$. Similarly,
$BB_{idom}$ is a immediate dominator for basic block $A$, if and only if,
$(BB_{idom} \in dom(A)) \land (\forall B \in dom(A), BB_{idom} \notin dom(B))$.
Finally, the dominator tree represents all basic blocks with tree nodes and adds
directed edges between them according to the immediate dominator relationship.

\paragraph{Dataflow analysis}
The algorithm is a classic forward dataflow analysis. In this paragraph, we will
quickly cover the key points in the algorithm. For more detailed information,
one should consult Cytron's paper on SSA construction. During pass
initialization, we first set the following,
$\forall A \in BB \setminus \{ BB_{entry} \}, dom(A) = BB$,
where $BB$ denotes the set of all basic blocks that appeared in the control flow
graph, and $BB_{entry}$ denotes the entry basic block for CFG. For the entry
basic block, we set $dom(BB_{entry}) = \{ BB_{entry} \}$ instead. The
initialization value is a conservative guess of the result, and the next step is
to refine it. The iterative step rule is as follow,
$$
    \forall A \in BB, dom(A) =
    \left\{\{ A \} \cup \bigcap_{B \in pred(A)} dom(B)\right\}
$$
Here, $pred(\cdot)$ denotes the predecessor of the given basic block. The
general idea is that for each of the basic blocks, a basic block that dominates
all its predecessors must also dominate the given basic block. The stop criteria
for the dominator analysis are also quite simple. If there are no more changes
in the result, the forward analysis will terminate.

\paragraph{Implementation}
SableWasm implements the forward dataflow analysis we discussed above with class
\texttt{DominatorPass}. In addition, the analysis pass object shares its result
map with a helper class \texttt{DominatorPassResult} which provides helper
methods for accessing the result, such as calculating the immediate dominator
and constructing the dominator tree from the result sets. Finally, SableWasm
uses several techniques to improve the performance, such as modelling the set
with sorted arrays.

In this section, we presented the dominator analysis in SableWasm. The dominator
analysis is quite common among compiler implementations, and it will play a
critical role in the latter part of the project.