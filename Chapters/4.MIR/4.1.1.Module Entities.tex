\section{MIR Module Entities}
\label{section:mir-design-module-entities}

\begin{figure}
  \centering
  \includegraphics[width=\textwidth]{Images/4.MIR/module.pdf}
  \caption{SableWasm MIR Module-level entities}
  \label{fig:sablewasm-mir-module}
\end{figure}

SableWasm module-level entities are the top-level elements in a translation
module. They are direct implementations of the WebAssembly module entities
defined in the specification. Figure~\ref{fig:sablewasm-mir-module} presents a
general illustration of the SableWasm module-level entities. In this section,
we will cover the design of each entity and compare it with its WebAssembly
correspondent. All SableWasm module-level entities can optionally have import
and export annotates, except \texttt{data} and \texttt{element}. These
annotations correspond to the import and export entries defined in the
WebAssembly specification.

\paragraph{Function}
In figure~\ref{fig:mir-fibonacci}, we have a function definition at line 8. A
function declaration in SableWasm provides information about the type, local
variables, and name. A function definition should satisfy all the function
declaration requirements and, in addition, provide a function body using basic
blocks. The design of the function declaration and definition in SableWasm is
quite similar to that of WebAssembly. The only major difference is how to
represent the function body. We will come back to this in the later sections
within this chapter when we discuses the design of SableWasm MIR instructions.

\paragraph{Global}
SableWasm's global variable declaration and definition follow the design in
WebAssembly. In SableWasm, we relax several of the constraints defined in the
WebAssembly specification and its extensions. In the SIMD extension proposal,
the 128-bit vector type, \texttt{v128}, is only suitable within the function
body. There is no direct way to pass a vector value to the host environment, as
there is a lack of standard representation for 128-bit packed vectors in
JavaScript \footnote{This might subject to change in the future.}. In SableWasm,
we treat all primitive types uniformly. Thus, a global variable can contain an
integral value, a floating-point value, or even a packed SIMD vector. The type
for the SableWasm MIR global variable follows the specification in WebAssembly;
it consists of a value type and a constness modifier. In
figure~\ref{fig:mir-fibonacci}, we have a global variable definition at line 6,
which introduces a mutable 32-bit integral value. All global variable
definitions in SableWasm must provide a value initialization via an initializer
expression. In SableWasm MIR, all initialization expressions are constant
expressions, meaning that the host system can deduce the resulting values at the
module initialization phase. At runtime, the host system will first evaluate
these expressions and then initialize the global variables accordingly.
We will come back to the initialization expressions in detail later in this
chapter.

\paragraph{Memory and Data}
Memory and Data are implementations of the WebAssembly linear memory and its
initializer, respectively. One might think that there is no need to separate the
memory initializer from the memory entity definition, as in WebAssembly
specification, all data section entries must provide a valid linear memory
index. In the early version of SableWasm, we indeed adopt such implementation.
However, this approach might be subject to a significant change in an extension
that might soon merge to the WebAssembly specification. The WebAssembly bulk
memory operation extension proposal \footnote{WebAssembly bulk memory
  operations: \\\url{https://github.com/WebAssembly/bulk-memory-operations}}
introduces new instructions, such as \texttt{memory.fill} that directly refers
to a data section segment. Moreover, the proposal relaxes the constraints on the
linear memory index. Now the index can behave as a flag indicating whether the
data segment itself is active or not and no longer serves as a linear memory
index. Hence, to make our framework `futureproof', we separate linear memory
declarations from their initializers. Figure~\ref{fig:mir-fibonacci} presents a
linear memory definition at line 2. SableWasm memory entities also adopt
WebAssembly's linear memory type. The type consists of a pair of unsigned
integers, indicating the lower bound and upper bound of the memory size in
WebAssembly pages. The example above defines a memory with a minimal size of 2
pages, 128KiB, and exports it under the name `memory'. It, however, does not
provide any example for data initializers, although they are quite easy to
understand: a data initializer is essentially a binary chunk with an
initialization offset, and is semantically equivalent to a data section entry
in an ELF file.

\paragraph{Table and Element}
SableWasm's table and element entity implement the indirect table and its
initializer, namely element segment, accordingly. They follow the same principle
as the memory and data entity in the previous section. Currently, like a data
segment entry, WebAssembly's element section entry must refer to a valid
indirect table via an index. In the future, this may also subject to change. The
WebAssembly reference types extension proposal \footnote{WebAssembly reference
  types: \url{https://github.com/WebAssembly/reference-types}} introduces
instructions such as \texttt{table.fill} that are able to have direct access to
element segment initializers. \texttt{table.fill} instruction is similar to
\texttt{memory.fill} defined in the bulk memory operation extension. It will
copy a sequence of compile-time defined function pointers into an indirect table
at runtime. Thus, when we design our table entity, we also split the
declarations from their initializers. The type for table entity is the same as
the table type in WebAssembly. It consists of a pair of unsigned integers,
indicating the lower bound and upper bound for the number of function pointers
stored in the indirect table. In SableWasm MIR, we treat memory entities and
table entities as black boxes, and its concrete implementation is deferred to
the backend. In the example shown in figure~\ref{fig:mir-fibonacci}, the module
defines a table entity at line 4 that stores exactly one function pointer. Note
that the table entity does not require users to initialize the value for all
entries. The table entity default initializes all entries to null pointers.

In this section, we covered the design for module-level entities in SableWasm.
They are pretty similar to the those defined in the WebAssembly specification.
In the next section, we will move the design of SableWasm initialization
expressions.