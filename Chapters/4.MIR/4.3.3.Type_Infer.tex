\subsection{Type Infer}

This section presents the type system for SableWasm MIR. SableWasm MIR is a statically typed language with a pretty straightforward type system. However, one may already notice that SableWasm MIR does not annotate every instruction with a type, unlike many other compiler intermediate representations. Instead, SableWasm computes the type for value on-demand via a set of type infer rules. The type system for SableWasm MIR generalizes from the MVP WebAssembly type system and its extension proposals with a few modifications. The formal definition for SableWasm MIR types are as follow,

\begin{lstlisting}[basicstyle=\linespread{1}\ttfamily, mathescape=true]

$\langle$primitive_type$\rangle$ ::= i32 | i64 | f32 | f64 | v128
$\langle$tuple_type$\rangle$     ::= (N, $\langle$primitive_type$\rangle$$\dots$)
$\langle$type$\rangle$           ::= $\langle$primitive_type$\rangle$ | <tuple_type$\rangle$ | () | $\bot$

\end{lstlisting}

Here we will skip the discussion for \emph{primitive type} and the type checking rules for its corresponding instructions as they are equivalent to the MVP WebAssembly type system. One should consult the specification for more details. The \emph{tuple type} consists of an unsigned integer and a list of primitive types. They model the return types of multi-value return functions or \texttt{Pack} instructions. Finally, we introduce the unit type, $()$ and bottom type, $\bot$. One can consider the unit type as \texttt{void} in the C programming language. They represent no value present, but the type is valid. On the other hand, the bottom type, $\bot$, signals that the pass can not assign any valid type to the term. In the rest of this section, we will focus on our discussion on extensions made to major WebAssembly extension proposals, multi-value and SIMD operation.

\paragraph{Multi-value} WebAssembly multi-value extensions allow functions to have more than one return values, which is quite interesting. Usually, low-level bytecode representation does not directly support this feature and usually only appears in higher-level language design, such as Python. In section 4.1.3, we introduced two instructions \texttt{Pack} and \texttt{Unpack}, along with how we represent multi-value for functions. As a quick recap, SableWasm uses tuple to denote the multi-value return for functions. \texttt{Pack} instruction collects values and constructs a tuple containing them, while on the other hand, \texttt{Unpack} extracts primitive values from tuples. Let's focus on the \texttt{Pack} instruction first. The typing rule for \texttt{Pack} is straightforward. If we can infer types for all candidate values, we say that the \texttt{Unpack} instruction has a tuple type consisting of the number of candidate values and a list of element types. On the other hand, if any of the candidate values result in a non-primitive type, the \texttt{Unpack} instruction is $\bot$ type. More formally,
$$
    \frac{\Gamma \vdash v_0 \Rightarrow t_0, \dots, v_n \Rightarrow t_n \qquad \forall i, t_i \in primitives}{\Gamma \vdash \text{\textbf{pack} } v_0, \dots, v_n \Rightarrow \langle n, t_0 \dots t_n \rangle}
    \qquad
    \frac{\Gamma \vdash \exists i, v_t \notin primitives}{\Gamma \vdash \text{\textbf{pack} } v_0, \dots, v_n \Rightarrow \bot}
$$
Here the set $primitives$ is the set of all possible primitive types in the SableWasm MIR type system. For \texttt{Unpack} instructions, the type checker will first check if the immediate index is within the tuple size. If the index is out of bounds, the type checker will assign the instruction with bottom type $\bot$. Otherwise, it will take the type from the tuple specified by the index. Formally,
$$
    \frac{\Gamma \vdash v \Rightarrow \langle n, t_0 \dots t_n \rangle \qquad k < n}{\Gamma \vdash \text{\textbf{unpack } k v} \Rightarrow t_k}
    \qquad
    \frac{\Gamma \vdash v \Rightarrow \langle n, t_0 \dots t_n \rangle \qquad k \geq n}{\Gamma \vdash \text{\textbf{unpack } k v} \Rightarrow \bot}
$$
We also generalize the function type in WebAssembly, and SableWasm MIR's function type will always have a single return value. We use the following strategy to map WebAssembly's function type into SableWasm MIR function type. In the case where there are no return values, we translate the return type into unit type. For example, SableWasm translate \texttt{[i32] -> []} into \texttt{[i32] -> ()}. On the other hand, if the function type has exactly one return value, the translation rule is trivial. Finally, when there are multiple return values, we pack them into a single tuple. For example, SableWasm use \texttt{[i32] -> (2, i32, f32)} to represent \texttt{[i32] -> [i32, f32]}  in WebAssembly.

\paragraph{SIMD operation}
Section 4.1.3 presents the instruction design in SableWasm MIR. We mentioned that WebAssembly's 128-bit vector value, added by the SIMD operation extension proposal, does not store their shape information in the type. Therefore, it is the instruction that is responsible for the instructions to interpret the shape correctly. WebAssembly's design gives us two choices in SableWasm when designing a type system for vector operations. First, we can follow the procedure in SableWasm, erase all the shape information for the value, and carefully plan the instruction semantics to make sure that all the operations have defined behaviour at runtime. Second, another approach is to add shape information back to the value. If there is a mismatch in shape formation, either the translation visitor can insert a bit cast, or the type checker can reject the program. In SableWasm MIR, we take the first approach by erasing all the shape information from the vector values. The later chapter on backend design will introduce the second approach in detail. The semantics for SIMD instruction in SableWasm MIR follows the WebAssembly's specification. We always store the value using the little-endian method, and the vectors start their first lane from the least significant bit.

In this section, we talk about the type infer pass in SableWasm MIR. Similar to the dominator analysis we seen in section 4.3.1, the type infer pass does not optimize the control-flow graph. But they are critical in the backend when we lower the SableWasm MIR into LLVM. We will come back to this in detail in chapter 5.