\subsection{MIR Initializer Expressions}

\begin{figure}
  \centering
  \includegraphics[width=0.85\textwidth]{Images/4.MIR/initalizer-expression.pdf}
  \caption{SableWasm MIR Initializer Expression}
  \label{fig:sablewasm-mir-initializer-expression}
\end{figure}

WebAssembly defines a particular form of expression for initialization, namely
constant expressions. They can appear in three locations in the current
specification. First, global variables declaration can contain constant
expression as their initialization values. Additionally, data section entries
and element section entries can have constant expressions as the offsets for
their initialization payload. In SableWasm MIR, we define initializer
expressions that act similar to what constant expressions do in WebAssembly.
Figure~\ref{fig:sablewasm-mir-initializer-expression} gives a general
illustration about SableWasm MIR initializer expressions. The initializer
expressions are quite simple. In the current WebAssembly and SableWasm, an
initializer expression can be either a constant value or refer to an imported
global via \texttt{GlobalGet} instruction. Hence, in principle, currently,
a SableWasm MIR initializer expression is essentially a single instruction. In
the future, one may generalize such constraints by allowing more complex
constructs in initializer expressions.

\paragraph{Constant}
The \texttt{Constant} instruction represents a single constant value for the
initializer expression. In WebAssembly, a constant value can be one of the
following: a 32-bit or 64-bit integer, a floating-pointer number, or a 128-bit
SIMD vector \footnote{With WebAssembly SIMD128 extension}, and the specification
encodes the type within the instruction opcode. Hence, there are multiple
instructions in WebAssembly to introduce a constant. In SableWasm, we do not
encode the type into the opcode, and \texttt{Constant} instruction is the only
instruction that takes care of the task. In figure~\ref{fig:mir-fibonacci}, we
have a constant initializer at line 6 that initializes the value of the global
to a 32-bit integer with a value that equals 66560. When querying the type of a
\texttt{Constant} instruction, SableWasm will infer it according to its payload
constant.

\paragraph{GlobalGet}
The \texttt{GlobalGet} instruction is exactly same as the WebAssembly's
\texttt{global.get} in terms of execution semantics. The WebAssembly
specification allows any initializer expression to refer to an imported
\footnote{This might subject to change in the future version of WebAssembly}
global value. As these values are initialized before entering the module,
reading their value is always valid during module initialization. The example
in figure~\ref{fig:mir-fibonacci} does not provide an example of
\texttt{GlobalGet} as an initializer expression, as they are less frequently
used compared to constant initializer expression, especially for global values.
However, in some ABI implementations, data section entries and element section
entries require reading from global values serving as base pointers. SableWasm
also infer the type for \texttt{GlobalGet} initializer expression in a similar
fashion as \texttt{Constant}. In this case, the type of instruction is the same
as the referred global variable without the `constant' modifier.

In this section, we covered the design and implementation of initializer
expressions in SableWasm. They are pretty simple in the current design. We will
now move to the next part in the SableWasm design, the MIR instructions.