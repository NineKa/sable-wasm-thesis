\chapter{Middle-level Intermediate Representation}

This chapter describes SableWasm's middle-level intermediate representation
(MIR), which has a critical role in the entire compilation pipeline. The MIR
acts as a middle ground between the WebAssembly bytecode frontend and various
possible backends. Currently SableWasm only implements one backend that utilizes
the LLVM compilation framework, but adding more backend support should not
require significant modification on the MIR. It also implements an analysis and
transformation framework where we perform several optimizations over the MIR. We
will first go over the overall design of the MIR, then later move to the
translation strategy we used to translate WebAssembly bytecode to MIR. Finally,
we will end the chapter with several analyses and transformations we
implemented.

\section{MIR Design}

In the previous chapters, we covered the design of WebAssembly bytecode. A quick
reminder, WebAssembly is a stack-based intermediate representation (IR) where
all instructions operate over an implicitly declared operand stack. There are
several advantages of a stack-based IR. Perhaps the most important one is its
portability. A stack-based IR makes fewer assumptions on the machine than a
register-based one. One can even provide an implementation for a hypothetical
machine with only one register. Another advantage is the code size. Experiments
show that, in general, a stack-based IR is smaller in size than its
corresponding registered version \cite{stack-and-register-vm}. When designing a
binary format that ships executables over the internet, the stack-based IR seems
to be a better choice for WebAssembly.

Nevertheless, there are no silver bullets: a stack-based IR design also has its
drawbacks. One of them is the difficulty faced when performing code analysis and
transformation over the module. As for each instruction, its operands implicitly
come from the stack; the value use-definition relationship between instructions
is not apparent to the analysis, and recovering such connection between
instructions from the IR is not a trivial task.

\begin{figure}
    \centering
    \lstinputlisting[
        language=SableWasmMIR,
        basicstyle=\linespread{0.8}\ttfamily,
        numbers=left
    ]{Code/4.MIR/fibonacci.mir}
    \caption{Fibonacci in translated SableWasm MIR}
    \label{fig:mir-fibonacci}
\end{figure}

On the other hand, we have the register-based intermediate representation,
commonly abstracted to assume an infinite number of registers and requiring a
register allocation algorithm to map them to actual, physical registers. For
each instruction in register-based IR, it has its operand encoded in the
instruction. Hence, the use-definition relationship will become explicit to the
analysis and transformation.

The main design goal for SableWasm MIR is to provide an analysis platform for
the entire compiler system. Thus, we implement our MIR as an infinite register
machine. We also take a traditional approach in various other aspects. For
example, instead of using the structured control flow similar to what
WebAssembly offers, we use \emph{control-flow graphs} (CFGs) to represent the
relationship between basic blocks. The SableWasm MIR is also in
\emph{single static assignment} (SSA) form \cite{ibm-ssa}, as covered in the
background chapter. The design for instruction and module-level entities in
SableWasm MIR is quite similar to what WebAssembly instruction offers. One can
view the SableWasm MIR as a mixture of the target LLVM intermediate
representation and the source WebAssembly bytecode. We also adopt several design
features from LLVM IR into MIR, such as automatically managed use-site lists,
which provide each AST node with an efficient way to access their use sites.
In SableWasm MIR, all elements are derived from the base class \texttt{ASTNode}
which implements these features that are helpful later in MIR analysis and
transformation.

Figure~\ref{fig:mir-fibonacci} shows a simple function that calculates Fibonacci
numbers with a recursive method in SableWasm MIR. With the help of the figure,
we will go through the detailed design of SableWasm later in the chapter. We
will first present the module-level entity design and their initializer
expressions, such as functions, then move to the design of each instruction
defined in MIR.

\subsection{MIR Module Entities}

\begin{figure}
  \centering
  \includegraphics[width=\textwidth]{Images/4.MIR/module.pdf}
  \caption{SableWasm MIR Module-level entities}
  \label{fig:sablewasm-mir-module}
\end{figure}

SableWasm module-level entities are the top-level elements in a translation module. They are direct implements of the WebAssembly module entities defined in the specification. Figure~\ref{fig:sablewasm-mir-module} presents a general illustration of the SableWasm module-level entities. In this section, we will cover the design of each entity and compare them with its WebAssembly correspondent.

\paragraph{Function}
In figure~\ref{fig:mir-fibonacci}, we have a function definition at line 8. A function declaration in SableWasm provides information about the type, local variables and name. A function definition should satisfy all the requirements of function declaration, and additionally, provides a function body using basic blocks. The design of the function declaration and definition in SableWasm is quite similar to that of WebAssembly. The major difference is how to represent the function body. We will come back to this in the later sections within the chapter. Finally, like other module-level entities, a SableWasm function can optionally have import or export annotations. These annotations provide names for the import and export entries in the WebAssembly module.

\paragraph{Global}
SableWasm's global variable declaration and definition follow the design in WebAssembly. In SableWasm, we relax several of the constrain defined in WebAssembly specification and its extensions. In the SIMD extension proposal, the 128-bit vector type is only suitable within the function body. There is no direct way to pass a vector value to the host environment, as there is a lack of standard representation for 128-bit packed vectors in JavaScript \footnote{This might subject to change in the future. WebAssembly SIMD extension proposal is still in the drafting process.}. In  SableWasm, we treat all primitive types uniformly. Thus, a global variable can contain an integral value, a floating-point value or even a packed SIMD vector. The type for the global variable follows the specification in WebAssembly; it is a pair of value type and constness modifier. In figure~\ref{fig:mir-fibonacci}, we have a global definition at line 6, which contains a mutable 32-bit integral value. All global variable definitions in SableWasm must provide a value initialization via initializer expression, which we covered in the previous section. The rules for import and export annotation on the SableWasm function entities also apply to SableWasm global variables, which we do not show in the example above.

\paragraph{Memory and data}
Memory and Data are implementation for WebAssembly linear memory and its initializer, respectively. One might think that there is no need to separate the memory initializer from the memory entity definition, as in WebAssembly specification, all data section entries must provide a valid linear memory index. In the early version of SableWasm, we indeed adopt such implementation. However, this approach might be subject to a significant change in an extension that might soon merge to the WebAssembly specification. The WebAssembly bulk memory operation extension proposal \footnote{WebAssembly bulk memory operations: \\\url{https://github.com/WebAssembly/bulk-memory-operations}} introduce new instructions, such as \texttt{memory.fill} that direct refers to a data section segment. Moreover, the proposal relaxes the constrain on the linear memory index. Now the index can behave like a flag indicating whether the data segment itself is active or not and no longer serves as a linear memory index. Hence, to make our framework `futureproof', we separate the linear memory declaration from their initializers. Figure~\ref{fig:mir-fibonacci} presents a linear memory definition at line 2. SableWasm memory entities also adopt WebAssembly linear memory type. The type consists of a pair of unsigned integers, indicating the lower bound and upper bound of the memory size in WebAssembly pages. Finally, the memory entity can have import and export annotations similar to other module-level entities in SableWasm. The example above defines a memory with a minimal size of 2 pages, 128KiB, and exports it under `memory'. The example above does not provide any example for data initializers, but they are quite easy to understand. A data initializer is essentially a binary chunk with an initialization offset. They are semantically equivalent to a data section entry in an ELF file.

\paragraph{Table and element} SableWasm table and element entity implements the indirect table and its initializer, namely element segment, accordingly. They follow the simple principle as the memory and data entity in the previous section. Currently, like data segment entry, WebAssembly's element section entry must refer to a valid indirect table via an index. In the future, this may also subject to change. WebAssembly reference types extension proposal \footnote{WebAssembly reference types: \url{https://github.com/WebAssembly/reference-types}} introduce instructions such as \texttt{table.fill} that are able to have direct access to element segment initializers. \texttt{table.fill} instruction is similar to \texttt{memory.fill} defined in the bulk memory operation extension. It will copy a sequence of compile-time defined function pointers into an indirect table at runtime. Thus, when we design our table entity, we also split the declarations from their initializers. The type for table entity is the same as the table type in WebAssembly. It consists of a pair of unsigned integers, indicating the lower bound and upper bound for the number of function pointers stored in the indirect table. In SabelWasm MIR, we treat memory entities and table entities as black boxes, and its concrete implementation is deferred to the backend. The example shown in figure~\ref{fig:mir-fibonacci}, the module defines a table entity at line 4 that stores exactly one function pointers. Note that the table entity does not require users to initialize the value for all entries. The table entity default initializes all entries to null pointers. Finally, the rule for import and export annotation also apply for table entity. However, the element entity is local to the module and can neither export nor import from other modules.

In this section, we cover the design for module-level entities in SableWasm. They are pretty similar to the sections defined in WebAssembly specification. In the next section, we will move the design of SableWasm instructions.
\subsection{MIR Initializer Expressions}

\begin{figure}
  \centering
  \includegraphics[width=0.85\textwidth]{Images/4.MIR/initalizer-expression.pdf}
  \caption{SableWasm MIR Initializer Expression}
  \label{fig:sablewasm-mir-initializer-expression}
\end{figure}

WebAssembly defines a particular form of expression for initialization, namely
constant expressions. They can appear in three locations in the current
specification. First, global variables declaration can contain constant
expression as their initialization values. Additionally, data section entries
and element section entries can have constant expressions as the offsets for
their initialization payload. In SableWasm MIR, we define initializer
expressions that act similar to what constant expressions do in WebAssembly.
Figure~\ref{fig:sablewasm-mir-initializer-expression} gives a general
illustration about SableWasm MIR initializer expressions. The initializer
expressions are quite simple. In the current WebAssembly and SableWasm, an
initializer expression can be either a constant value or refer to an imported
global via \texttt{GlobalGet} instruction. Hence, in principle, currently,
a SableWasm MIR initializer expression is essentially a single instruction. In
the future, one may generalize such constraints by allowing more complex
constructs in initializer expressions.

\paragraph{Constant}
The \texttt{Constant} instruction represents a single constant value for the
initializer expression. In WebAssembly, a constant value can be one of the
following: a 32-bit or 64-bit integer, a floating-pointer number, or a 128-bit
SIMD vector \footnote{With WebAssembly SIMD128 extension}, and the specification
encodes the type within the instruction opcode. Hence, there are multiple
instructions in WebAssembly to introduce a constant. In SableWasm, we do not
encode the type into the opcode, and \texttt{Constant} instruction is the only
instruction that takes care of the task. In figure~\ref{fig:mir-fibonacci}, we
have a constant initializer at line 6 that initializes the value of the global
to a 32-bit integer with a value that equals 66560. When querying the type of a
\texttt{Constant} instruction, SableWasm will infer it according to its payload
constant.

\paragraph{GlobalGet}
The \texttt{GlobalGet} instruction is exactly same as the WebAssembly's
\texttt{global.get} in terms of execution semantics. The WebAssembly
specification allows any initializer expression to refer to an imported
\footnote{This might subject to change in the future version of WebAssembly}
global value. As these values are initialized before entering the module,
reading their value is always valid during module initialization. The example
in figure~\ref{fig:mir-fibonacci} does not provide an example of
\texttt{GlobalGet} as an initializer expression, as they are less frequently
used compared to constant initializer expression, especially for global values.
However, in some ABI implementations, data section entries and element section
entries require reading from global values serving as base pointers. SableWasm
also infer the type for \texttt{GlobalGet} initializer expression in a similar
fashion as \texttt{Constant}. In this case, the type of instruction is the same
as the referred global variable without the `constant' modifier.

In this section, we covered the design and implementation of initializer
expressions in SableWasm. They are pretty simple in the current design. We will
now move to the next part in the SableWasm design, the MIR instructions.
\subsection{MIR Instructions}

SableWasm MIR uses control-flow-graph (CFG) based representations in static-single-assignment (SSA) form to represent code body in function definitions. We have provided an introduction to CFG and SSA in the background chapter. Here is a quick recap. CFG splits the control flow within the function into basic blocks. A basic block represents the most extended instruction sequence without control flow transfer, such as branching. Note that for function calls, we take a similar approach that LLVM adopted. We will come back to this in detail later in the section. Additionally, SSA requires that all values must have a unique definition site. Hence, in SSA form, the use-def chain is trivial to compute, while in a traditional CFG, one would need to extract from the graph with the help of reaching definition analysis. SableWasm instruction set is similar to WebAssembly bytecode in terms of semantics for most of the instructions. However, it operates over an infinite register machine instead of a stack-based machine. We also want to keep the size of the SableWasm instruction minimal. Hence, some of the instruction has different semantics compared to the counterpart appears in the WebAssembly specification. In this section, we will cover the design and implementation of SableWasm instructions. The following section will cover the translation strategy between WebAssembly bytecode and the SableWasm instruction set and instruction reduction rules.  Figure~\ref{fig:sablewasm-mir-inst} provides a general illustration of the design of the SableWasm instruction set. The SableWasm instruction sets can currently cover all the instructions set defined in WebAssembly specification, along with several extensions such as multivalue and SIMD vector operations.

\begin{figure}
  \centering
  \includegraphics[width=0.8\textwidth]{Images/4.MIR/sablewasm-instruction.pdf}
  \caption{SableWasm MIR Instructions}
  \label{fig:sablewasm-mir-inst}
\end{figure}

\paragraph{Terminating instructions}
As we discussed above, SableWasm split the function control flow into basic blocks, which contain the maximum amount of consecutive instructions without control flow transfer. In addition, SableWasm, similar to many other SSA form instruction sets, defines a particular group of instructions called terminating instructions. These instructions signal a control flow transfer out of the current basic block, and they must only appear as the last instruction in any given basic block. SableWasm defines four different terminating instructions: unreachable, unconditional branching, conditional branching and table branching. If the control flow reaches a \texttt{unreachable} instruction, the runtime system will signal a runtime panic. The \texttt{unreachable} instruction in SableWasm is identical to its counterpart in WebAssembly in terms of semantics. The \texttt{Unconditonal} instruction is an unconditional control flow transfer, as the name suggests. It refers to a basic block as the operand. At runtime, the instruction will always transfer the control flow to the target basic block. \texttt{Unconditional} is similar to the \texttt{br} instruction defined in WebAssembly specification. On the other hand, \texttt{Conditional} is a conditional branching. It takes a value and two basic blocks as its operands. At runtime, the instruction will compare the value against integral value zero. If the value equals zero, the instruction will transfer the control flow to the `false' basic block, otherwise, to the `true' basic block. SableWasm's \texttt{Conditional} instruction is similar to \texttt{br.cond} defined in WebAssembly. The last terminating defined in SableWasm is \texttt{Switch}. \texttt{Switch} instruction is comparable to the \texttt{br.table} instruction in WebAssembly. The instruction takes a value, a list of basic blocks, and a default branching basic block as its operands. At runtime, \texttt{Switch} will interpret the value as an integral value and dispatch accordingly. If the value is within the branching list's range, it will redirect the control flow to the basic block referred to by the index. Otherwise, \texttt{Switch} will transfer the control flow to the default basic block.

\paragraph{Function call}
In SableWasm, we provide two instructions for function calls defined in WebAssembly specification, direct function calls and indirect function calls. \texttt{Call} defines a direct function call where the callee is known at compile time. It takes a function as callee and a list of arguments as operands. On the other hand, \texttt{CallIndirect} defines an indirect function call. It implements the indirect function call protocol defined in WebAssembly specification. A quick reminder, in WebAssembly, an indirect function call takes an indirect table, the table index, the expecting function type and a list of values as arguments. At runtime, the system should first check if the index is valid for the indirect table and fetch the function pointer and its actual signature accordingly. Then, the system should compare the signature against the expecting type. If the signature matches the type, the runtime system will transfer the control flow to the function referred to by the function pointer. Implementing the signature verification mechanism is backend-specific; we will return to this topic in the next chapter. Note that we do not treat function call instructions as terminating instructions, even transferring the control flow to other locations. In SableWasm MIR, we follow the design that appeared in LLVM intermediate representation. It is guaranteed that the control flow will continue to the next instruction for function call instructions. Hence, from the basic block's local point of view, their control flow is pre-determined, and there is no difference compared to other non-terminating instructions.

\paragraph{Local and global variable access}
In WebAssembly, instructions have access to locals defined by its parent function and global variables defined by its enclosing module. SableWasm defines getter and setter instruction for both local and global variables to implement the specification. Their semantics are the same compare to WebAssembly's counterparts. We will skip the detail here, but one can consult the WebAssembly specification for detailed information.

\paragraph{Numerical operations}
In SableWasm, we classify the numerical operations into three different categories, the \texttt{Compare} instructions, \texttt{Unary} instructions, and \texttt{Binary} instructions. The \texttt{Compare} instructions implements the comparison between values, such as `equal to'. They always yield a 32-bit integer as WebAssembly specification suggests. The \texttt{Unary} and \texttt{Binary} implements , as their name suggests, unary and binary operations between values. The result of \texttt{Unary} and \texttt{Binary} instruction is dependent on the opcode. On the other hand, we can also orthogonally classify the instructions into integer, floating-point and packed integer and packed floating numbers. Note that in MVP WebAssembly, there are only integer and floating-point value operations; the SIMD operation extension proposal adds the packed value operation to the instruction set. In the WebAssembly SIMD extension proposal, the vector value does not store its shape information in the types. Instead, the packed value instructions' opcodes keep track of the shape of the vector values, which leads to the blot of instruction opcodes. In SableWasm, we separate the instruction opcode from the vector shape. For each of the packed value operations, it must have either a \texttt{SIMD128IntLaneInfo} or \texttt{SIMD128FPLaneInfo}. Figure~\ref{fig:sablewasm-mir-inst} shows all the class of numerical operations defined in SableWasm. For detailed opcode of each numerical instruction class, we include them in the thesis appendix.

\paragraph{Load and store}
\texttt{Load} and \texttt{Store} instruction provides access to the linear memory for SableWasm MIR. Although in the current version of WebAssembly, the module can contain at most one linear memory. All load and store implicitly refer to this linear memory \footnote{This subject to change in the future version of WebAssembly.}. \texttt{Load} instruction takes a linear memory and an integer value as operands. At runtime, the value will treat as the address (or offset) respective to the start of the linear memory, and the instruction yields to fetched result. In WebAssembly, the \texttt{load} instruction associates with a type and an extension method. For example, \texttt{i32.load8\_s} load an 8-bit integer from the linear memory, and then signed extends the fetched byte into a 32-bit integer. In SableWasm, \texttt{Load} instruction only associates to an integer value, namely the load width. The load width must equal to or smaller than the load type's width. Also, SableWasm \texttt{Load} always perform zero-extension on loaded value. Hence, when translating WebAssembly's sign-extended load into SableWasm's \texttt{Load}, one must combine the load instruction with a cast instruction. We will come back to this later in the chapter. \texttt{Store} instruction also associate with a store width. Like the load width defined for \texttt{Load} instruction, store width must also be equal to smaller than the store value type's width. At runtime, the system will first perform bit truncate to the value and then store the result into the linear memory. One may notice that in SableWasm, we erase the alignment attribute and offset attribute defined in WebAssembly. Currently, we do not support alignment hints from the WebAssembly module. In SableWasm, the load and store always have the alignment requirement of one byte. This implies that the load and store can happen anywhere in the linear memory, which corresponds to WebAssembly's linear memory specification.

\paragraph{Linear memory manipulation}
WebAsseembly specification defines three instruction can manipulate linear memories, such as \texttt{memory.size}, \texttt{memory.grow}. Like the \texttt{load} instruction we covered in the previous paragraph, all these instructions operate over the implicitly defined unique linear memory within the module. In SableWasm, we provide similar \texttt{MemoryGrow} and \texttt{MemorySize} instruction. The semantics of the SableWasm's memory manipulation instruction is the same as their WebAssembly counterparts, except that the linear memory needs to be explicitly stated. In SableWasm, we introduce a special instruction, \texttt{MemoryGuard} which is an explicit memory boundary check. In WebAssembly, all \texttt{load} and \texttt{store} instruction need to check for linear memory out of bound error before access. SableWasm separates the bound check from the memory access. One advantage of this is that one may implement static memory bounds check elimination optimization over SableWasm MIR. Additionally, one backend may provide different strategies for handling boundary checks, such as utilizing invalid virtual memory pages with the operating system's help. In this case, we only need to modify the translation pattern for \texttt{MemoryGuard}. \texttt{MemoryGuard} takes a linear memory and an integer value as the operand. It also associates with an integer immediate, known as guard width. At runtime, the system will perform a boundary check over the linear memory starting from the given address to determine if it contains at least a given number of bytes ahead. If there are not enough bytes available, the system should signal a runtime panic.

\paragraph{Pack and Unpack}
WebAssembly multivalue specification \footnote{WebAssembly Multi-value Proposal: \url{https://github.com/WebAssembly/multi-value}} relaxes the constrains on the function type. Functions now can return multiple values instead of at most one value. To support these features, we introduce \texttt{Pack} and \texttt{Unpack} instructions, along with extending WebAssembly's type system. \texttt{Pack} instructions group multiple values into an ordered tuple, while the \texttt{Unpack} reverse the operation by retrieving the value from tuples by index. In the case where a function returns multiple values, we use a tuple instead. SableWasm treats tuples as a first-class values; however, currently, tuples cannot be recursive. We will come back to this later in the chapter when we visit the type systems of SableWasm MIR. The index of the \texttt{Unpack} must be an immediate value in the current version of SableWasm MIR and is verified at compile time.

\paragraph{Vector operations}
In the previous paragraph, we introduce the numeric operations defined in SableWasm MIR. However, several instructions does not fix into neither \texttt{Unary} nor \texttt{Binary} instructions. Hence, to faithfully support the SIMD operations introduced by the extension proposal, we add four vector-specific operations into SableWasm MIR. They are \texttt{VectorSplat}, \texttt{VectorExtract}, \texttt{VectorInsert} and \texttt{VectorByteShuffle}. \texttt{VectorSplat} will broadcast the operand value to all lanes in the result vector. SableWasm MIR defines vector splat operation for both packed integer vector and packed floating-point vector. \texttt{VectorExtract} is similar to the \texttt{extractelement} defined in LLVM intermediate representation. It takes a vector as the operand and also associates itself with an immediate integer value. At runtime, the system extracts the value of the given lane and yields as a result. \texttt{VectorInsert} is similar to \texttt{insertelement} defined in LLVM. It will replace the vector operand with a given value and yields the updated vector as a result. Note that in the WebAssembly SIMD extension proposal, there are more instructions defined that modify the individual lane value of the vector, such as \texttt{V128Load32Lane} which loads a 32-bit value into a specific lane within the vector. In this project, we would like to keep our instruction set simple; hence, these instructions are reduced into multiple SableWasm MIR instructions. We will come back to this later in the chapter when we discuss the instruction reduction rules. The last instruction we introduced is the \texttt{VectorByteShuffle}. \texttt{VectorByteShuffle} is similar to \texttt{shufflevector} defined in LLVM, except that it operates on bytes instead of lanes. Currently, the \texttt{VectorByteShuffle} only operates over an array of immediate integer values. Compare to the lane shuffle semantics, byte shuffle semantics provides more precise control over the result value. One can trivially simulate the lane shuffle with byte shuffle. The WebAssembly SIMD extension proposal only defines shuffle for \texttt{i8x16}, which corresponding to the byte shuffle semantics. However, in the future, if another shape vector supports shuffle operation, one can generalize the implementation with minimal modification.

\paragraph{Cast}
\texttt{Cast} models the conversion of values to their equivalent form in other types. One may argue that \texttt{Cast} instruction is just a numerical \texttt{Unary} operations, and it is partially true. However, in SableWasm MIR, we would like to group the conversion and extension instructions into their groups; and later in the analysis phase, we can focus on numerical operations in the case of \texttt{Unary}.  In SableWasm MIR, we do not distinguish between value conversion and value extension. We treat signed and zero extensions as a kind of value conversion. The \texttt{Cast} instruction takes a single value as the operand, and it associates itself with a cast opcode. At runtime, it will perform the conversion according to the opcode, and if the result cannot be accurately represented in the target type, the system should signal an error. The cast opcodes are direct implementations of their WebAssembly counterparts, and we will skip the detail here. One may refer WebAssembly specification for more details.

\paragraph{Intrinsic}
The last SableWasm MIR instruction we are going to cover in this chapter is the \texttt{Intrinsic} instructions. Most WebAssembly instructions can be represented by using the SableWasm MIR instructions, which we covered earlier in the section. However, there are still several corner cases. For example, the WebAssembly SIMD extension proposal defines Q-format rounding multiplication, a type of fix-point multiplication, for packed 16-bit integers. Another example is the \texttt{swizzle} operation. A \texttt{swizzle} operation is similar to a shuffle operation, except that it takes another vector as the shuffle indices vector instead of an array of immediate integer values. These operations are only defined for a specific vector shape and will introduce unneeded complexity to the SableWasm MIR if we generalize them to all possible vector shapes. Hence, here we group these instructions as the \texttt{Intrinsic} instructions. There is no direct mapping to LLVM instruction, even with the intrinsic functions provided by the framework for most of them. Hence, the backend is encouraged to support these instructions with runtime library routines.

In this section, we discussed the design of the SableWasm MIR instruction set, and in the next section, we will move the translation strategy between WebAssembly and SableWasm MIR.

\section{Translating WebAssembly to MIR}

In this section, we will cover the translation between WebAssembly bytecode and SableWasm MIR. We have covered the design of SableWasm MIR instructions previously. One may notice that for most of the instructions, especially for the numerical operations, SableWasm MIR shares the same semantics as WebAssembly. Hence, the translation rules for these instructions are pretty trivial, and we will not cover them in detail in this section. Instead, this section will focus on the translation rules for the structured control flow construct and WebAssembly instructions that require reduction during translation.

The translation framework is similar to the validation framework we discussed in the previous chapter except for two significant differences. First, the operation stack will keep track of the values generated during translation instead of types. Let's take \texttt{i32.add} as an example. \texttt{i32.add} instruction takes two values from the stack and then performs 32-bit integer addition between them. Finally, it will push the result type back onto the stack. In the case of translation stack, we pop two values from the stack and assume they are 32-bit integers. In SableWasm MIR, we use pointers to instructions to refer to the values generated by them. Then, the translation visitor will build a \texttt{IntBinaryOp} instruction, with \texttt{Add} as opcode, and two pointers as operands. Finally, the visitor will append the instruction to the current active basic block and push the instruction pointer of \texttt{IntBinaryOp} onto the stack. Another difference between the validation framework and the translation framework is the labels stack. In the validation framework, the labels stack stores the resulting types generated from the label. In the translation framework, the labels stack stores the potential landing basic blocks for the labels. We will revisit this with more details later in this section.

\subsection{Structured-Control-Flow Construct}

Translating from stack-based IR to register-based IR is not trivial, especially
when non-linear control flow structures appeared. This problem appeared in many
runtime system implementations, such as Numba \cite{numba}, a just-in-time (JIT)
compiler for Python. Usually, one needs some algorithm to recover the control
flow structure from annoying jump instructions. Luckily, in WebAssembly, we can
translate the stack-based bytecode into register-based basic blocks in linear
time, thanks to the structured-control-flow constructs and their validation
rules defined in WebAssembly. In this section, we will cover the translation
pattern used for WebAssembly's structured-control-flow constructs, namely
\texttt{block}, \texttt{if} and \texttt{loop}.

\begin{figure}
  \centering
  \includegraphics[width=\textwidth]{Images/4.MIR/translate-block.pdf}
  \caption{WebAssembly \texttt{block} translation pattern}
  \label{fig:translate-block}
\end{figure}

\begin{figure}
  \centering
  \includegraphics[width=\textwidth]{Images/4.MIR/translate-if.pdf}
  \caption{WebAssembly \texttt{if} translation pattern}
  \label{fig:translate-if}
\end{figure}

\paragraph{Block}
In the background chapter, we provide a general illustration of the three
structured-control-flow constructs. As a quick recap, \texttt{block} is the
simplest form of a structured-control-flow construct. It implicitly introduces
a label at the end of its enclosing instructions. A branching instruction
referring to this label will redirect the control flow to the end of the block.
Figure~\ref{fig:translate-block} illustrates the translation pattern for
WebAssembly \texttt{block} in SableWasm MIR. We will first clarify some of the
terminologies we used in the figure, and we will use the same terms later in
the \texttt{loop} and \texttt{if} pattern discussion. \emph{Expr Insert Point}
refer to the starting position for the generated instructions when we
recursively translate the instructions within the enclosing expression of the
\texttt{block} instruction. Furthermore, \emph{Label Insert Point} refer to the
position for generated instruction when we finish the recursive translation and
resume to the parent expression of the \texttt{block} instruction. A \emph{label
  stack entry} is a tuple consisting of a pointer to the landing BB, a list of
$\phi$ nodes expecting merge values, and a pointer to the \emph{label insert
  point}. The translation pattern for \texttt{block} is pretty simple; we
continue on the current BB and prepare the landing BB for the block instruction
as a branch instructions within the expression may refer to the label.
Additionally, to fully support multi-value extension in WebAssembly, we also
need to prepare the $\phi$ nodes in the landing BB. SableWasm generates the
$\phi$ nodes based on the type of the \texttt{block} instruction. WebAssembly
validation ensures that the expression within the \texttt{block} can access
exactly $m$ values from the stack and put $n$ values onto the stack. Finally,
we will append an unconditional branch to the landing BB because in WebAssembly,
if the control flow reaches the bottom of the \texttt{block} expressions, it
will implicitly fall through. For the operand stack, we will first pop $m$
values from the stack as \texttt{block} instruction's type suggests and push the
$\phi$ nodes as the result values. Now we need to set up the operand stack for
our the expression contained within the block. Again, due to the WebAssembly
validation rule, we need to insert a boundary before continuing.
Figure~\ref{fig:translate-block} represents this with the bold line in the
result operand value stack.


\paragraph{If}
The next control-flow structure defined WebAssembly is \texttt{if}.
WebAssembly's \texttt{if} is an expression instead of a statement that appears
in many other languages such as C. The \texttt{if} expression can yield some
values indicated by its type. Figure~\ref{fig:translate-if} illustrates the
translation patterns in SableWasm. There are two types of \texttt{if}
instruction defined in WebAssembly specification. The first case is a `partial'
\texttt{if} instruction, where it only contains the `true' branch. From
WebAssembly validation rules, it's easy to show that the only possible type is
\texttt{[i32]->[]}, even with the multi-value extension proposal. This implies
that the expression within the \texttt{if} instruction must start with an empty
operand stack. Hence, the translation pattern for the partial \texttt{if} is
quite straightforward: we only need to pop the condition value from the operand
stack and construct a conditional branch based on this value in the current BB.
On the other hand, we also have `full' \texttt{if} instructions with both `true'
expression and `false' expression. The validation rules ensure that both
expressions must have the same type. The translation pattern is more complex
compare to that of a `partial' \texttt{if}. In this case, we have to prepare the
landing BB similarly to what we did for the \texttt{block} construct. We need to
generate $n$ $\phi$ nodes for data-flow mergers from the true branch, the false
branch, and any possible branching instruction within both nested expressions.
Similarly, we need to pop $m$ values from the stack for operand values stack and
then push $n$ $\phi$ nodes. And, within both nested expressions, push $m$ values
back to the stack.

\begin{figure}
  \centering
  \includegraphics[width=\textwidth]{Images/4.MIR/translate-loop.pdf}
  \caption{WebAssembly \texttt{loop} translation pattern}
  \label{fig:translate-loop}
\end{figure}

\paragraph{Loop}
The last control-flow structure defined in WebAssembly is \texttt{loop}.
Figure~\ref{fig:translate-loop} gives a general illustration of SableWasm's
translation pattern for \texttt{loop} instructions. Similar to the `partial'
\texttt{if} we discussed in the previous paragraph, one can show that, under
WebAssembly's validation rules, the parameter types for the \texttt{loop}
instruction must equal to the result types. The \texttt{loop} instruction is
similar to the \texttt{block} instruction, except that if any branching
instruction refers to it, the branching instruction should transfer the control
flow to the start of the expression within the instruction instead of the end.
Thus, we need to prepare a standalone basic block for the nested expression in
\texttt{loop}, along with the $\phi$ nodes to merge value on each loop
iteration. Note that we also introduce $\phi$ nodes in the landing BB. One may
argue that there is no need for these $\phi$ nodes, as only one block can reach
the loop exit and no value merging will occur. Indeed, these $\phi$ nodes will
always be trivial $\phi$ nodes, which have only one possible value inflow.
However, this is due to the limitation of our translation framework.

In this section, we discussed the translation patterns for WebAssembly
structured control-flow constructs. Thanks to WebAssembly validation rules, the
types for these structured control-flow instructions explicitly mark value
merging and imply possible $\phi$ nodes. Furthermore, one can show that the
control graph generated above is indeed in SSA form. However, the directly
generated control flow graph is not easily understandable by users. This mainly
comes from two facts. First, the WebAssembly-targeting compiler may generate
awkward patterns to fit in the structured control-flow constructs. Second,
SableWasm translation patterns for structured-control flow constructs are not
optimal.
\subsection{Instructions Reduction}

This section will cover the instructions reduction rules used when lowering WebAssembly bytecode to SableWasm MIR. In the background chapter, we mentioned that one of WebAssembly's design goals is to be as compact as possible. When the community design the WebAssembly instruction set, they fuse several typical instruction sequences into single instructions. For example, SIMD vector operation extension defines \texttt{v128.load8x8\_s} which first load 8 single-byte integers into a vector, and then sign-extended them into 16 bit integers. Another example will be \texttt{v128.load32\_lane} which loads a 32-bit value, either a 32-bit integer or a single-precision floating-point number into a given vector. Such design is understandable for WebAssembly as binary size do matter when shipping application over the internet. But, for SableWasm, a static compiler, we focus more on the size of the instruction set instead of the size of the intermediate representation. It is harder to write analysis for a bloated instruction set, as one needs to consider more instruction cases. Hence, when lowering WebAssembly bytecode to SableWasm MIR, we replace some WebAssembly instructions with a sequence of SableWasm MIR instructions.


\paragraph{Eqz} \quad
\begin{lstlisting}[basicstyle=\linespread{1}\small\ttfamily, language=SableWasmMIR, mathescape=true]
[..., %n i32] i32.eqz $\Longrightarrow$ %t0 = i32.const 0; %t1 = int.eq %n %t0
[..., %n i64] i64.eqz $\Longrightarrow$ %t0 = i64.const 0; %t1 = int.eq %n %t0
\end{lstlisting}
WebAssembly defines unary \texttt{eqz} operations for all integer values. As the name suggests, \texttt{eqz} compares the operand value against zero and yields to one if true, zero otherwise. In SableWasm MIR, we group all comparison instructions into \texttt{Compare} class, and \texttt{eqz} does not fit into the class as it is not a binary operation. Hence we rewrite the \texttt{eqz} as \texttt{Compare} instruction with opcode as \texttt{Eq}.

\paragraph{Load} \quad
\begin{lstlisting}[basicstyle=\linespread{1}\small\ttfamily, language=SableWasmMIR, mathescape=true]
[..., %base i32] i32.load offset=%offset align=%align $\Longrightarrow$
    %addr = int.add %base %offset
    memory.guard %mem %addr 4
    %t0 = load.32 i32 %mem %addr
[..., %base i32] i32.load16_s offset=%offset align=%align $\Longrightarrow$
    %addr = int.add %base %offset
    memory.guard %mem %addr 2
    %t0 = load.16 i32 %mem %addr
    %t1 = cast i32.extend.16.s %t0
[..., %base i32] i32.load16_u offset=%offset align=%align $\Longrightarrow$
    %addr = int.add %base %offset
    memory.guard %mem %addr 2
    %t0 = load.16 i32 %mem %addr
\end{lstlisting}
In SableWasm instruction design section, we introduce the \texttt{Load} and \texttt{MemoryGuard} in SableWasm MIR. A quick recap, SableWasm MIR \texttt{Load} instruction, compare to its WebAssembly counterpart, assumes access is in-bound, does not support offset attribute and always performs zero-extension on partial load. Hence, to properly support WebAssembly's \texttt{load} instructions, we need to reduce them with the strategy shown above. For load instructions that do not require an extension, such as \texttt{i32.load}, we first calculate the actual starting address, perform a memory boundary check with \texttt{MemoryGuard}, and then perform the memory read. On the other hand, for a partial load operation, we need first to perform the load operation using the same protocol as a normal load. Then, if the signed extension is needed, we add its corresponding cast instruction. In the example above, we demonstrate this with WebAssembly's \texttt{i32.load16\_s}. In this case, SableWasm append a \texttt{Cast} instruction with opcode \texttt{i32.extend.16} after the load operation.

\paragraph{Store} \quad
\begin{lstlisting}[basicstyle=\linespread{1}\small\ttfamily, language=SableWasmMIR, mathescape=true]
[..., %base i32, %val i64] i64.store offset=%offset align=%align $\Longrightarrow$
    %addr = int.add %base %offset
    memory.guard %mem %addr 8
    store.64 %mem %addr %val
[..., %base i32, %val i64] i64.store16 offset=%offset align=%align $\Longrightarrow$
    %addr = int.add %base %offset
    memory.guard %mem %addr 2
    store.16 %mem %addr %val
\end{lstlisting}
Similar to the \texttt{Load} instruction we discussed earlier. \texttt{Store} instruction also assume the memory access is always in range and does not provide the offset attribute. However, \texttt{Store} instruction will always perform truncation instead of extension. Further, the only possible truncation is the bit-truncation by discarding bits starting from the most significant bit. The instruction reduction rules for WebAssembly \texttt{store} instructions is similar to those for \texttt{load} instructions. In the example above, we demonstrate the rules with \texttt{i64.load} and its partial store version, \texttt{i64.store16} which only stores the lowest two bytes into linear memory. SableWasm insert \texttt{MemoryGuard} instructions in a similar fashion comparing to \texttt{load} instructions. Note that we do not insert explicit \texttt{Cast} instruction to perform the truncation. \texttt{Store} instruction will implicitly truncate the value according to the store width; in this case, it will truncate the 64-bit integer into a 16-bit integer.

\paragraph{SIMD extension proposal reduction rules} \quad
\begin{lstlisting}[basicstyle=\linespread{1}\small\ttfamily, language=SableWasmMIR, mathescape=true]
[..., %lhs v128, %rhs v128] v128.andnot $\Longrightarrow$
    %t0 = v128.not %rhs 
    %t1 = v128.and %lhs %t0
[..., %lhs v128, %rhs v128] i16x8.extmul_low_i8x16_s $\Longrightarrow$
    %t0 = cast i16x8.extend.low.i8x16.s %lhs
    %t1 = cast i16x8.extend.low.i8x16.s %rhs
    %t2 = v128.int.mul i16x8 %t0 %t1
[..., %lhs v128, %rhs v128] i16x8.extmul_low_i8x16_u $\Longrightarrow$
    %t0 = cast i16x8.extend.low.i8x16.u %lhs
    %t1 = cast i16x8.extend.low.i8x16.u %rhs
    %t2 = v128.int.mul i16x8 %t0 %t1
\end{lstlisting}
SIMD extension proposal introduces approximately 240 instructions into the WebAssembly instruction set. However, not all of them are simple single operation instructions. SIMD extension proposal also followqian xiqian xiqian xis WebAssembly's design goal, to ensure compactness of the generated program. The proposal presents reduction rules for several SIMD operation instructions, and in SableWasm, we take advantage of them to reduce the size of the instruction set. The first applicable instruction is \texttt{andnot} operation for vectors. The \texttt{andnot} is equivalent to perform bitwise on the right-hand-side operand, and then bitwise `and' operation between the left-hand-side operand and the temporary result. SableWasm reduce \texttt{andnot} into a pair of \texttt{not} instruction and \texttt{and} operation, as shown in the example above. The second group of reducible instructions is the \texttt{ExtMul} instructions. SIMD extension proposal defines \texttt{ExtMul} for all packed integer vectors except packed 64-bit integers. They are equivalent first to widen the vector using the appropriate extension and then multiply two operands. In the example above, we demonstrate with \texttt{i16x8.extmul\_low\_i8x16\_s} which perform \texttt{ExtMul} operation for packed 8-bit integers. SableWasm implements this instruction by first perform a signed extension on the lower half of the vector and multiply the temporary result as shown above. SableWasm also apply similar procedure to \texttt{i16x8.extmul\_low\_i8x16\_u}, except that it use zero-extension in \texttt{Cast} instruction instead of signed-extension.

\paragraph{SIMD load with zero-padding} \quad
\begin{lstlisting}[basicstyle=\linespread{1}\small\ttfamily, language=SableWasmMIR, mathescape=true]
[..., %base i32] v128.load32_zero offset=%offset align=%align $\Longrightarrow$
    %addr = int.add %base %offset
    memory.guard %mem %addr 4
    %t1 = load.32 i32 %mem %addr
    %t2 = const v128 0
    %t3 = v128.int.insert i32x4 0 %t2 %t1
\end{lstlisting}
WebAssembly SIMD extension proposal also introduces many variations of load operations. The first variation is the `zero-padding' load operation. The `zero-padding' load is equivalent to load a scalar from the linear memory and then insert it into a zero-initialized vector. We demonstrate this with the example above. In SableWasm MIR, we first load a scalar using the protocol we discussed above to load the 32-bit integer. Then, we insert it into a zero vector using \texttt{VectorInsert} instruction. The WebAssembly SIMD extension proposal defines `zero-padding' load operations for all packed integers and packed float-point numbers. The reduction rule for instructions with other vector shapes is similar to the pattern above.

\paragraph{SIMD load and splat} \quad
\begin{lstlisting}[basicstyle=\linespread{1}\small\ttfamily, language=SableWasmMIR, mathescape=true]
[..., %base i32] v128.load32_splat offset=%offset align=%align $\Longrightarrow$
    %addr = int.add %base %offset
    memory.guard %mem %addr 4
    %t1 = load.32 i32 %mem %addr
    %t2 = v128.int.splat i32x4 0 %t1
\end{lstlisting}
The second variation of SIMD vector load is the `load-and-splat' load operation. This type of load operation is a combination of scalar load operation and vector splat operation. It first loads a scalar from the linear memory and then broadcasts the value to all vector lanes. SableWasm uses a similar reduce rule compare to the `zero-padding' load operation, except that instead of inserting the scalar into a zero-initialized vector, we use \texttt{VectorSplat} to broadcast it. The example above demonstrate this with \texttt{v128.load32\_splat}. Similar to the `zero-padding' load operation, `load-and-splat' is defined for all packed integers and packed float-point numbers.

\paragraph{SIMD load lane} \quad
\begin{lstlisting}[basicstyle=\linespread{1}\small\ttfamily, language=SableWasmMIR, mathescape=true]
[..., %base i32, %vec v128]
v128.load32_lane offset=%offset align=%align lane=%lane $\Longrightarrow$
    %addr = int.add %base %offset
    memory.guard %mem %addr 4
    %t1 = load.32 i32 %mem %addr
    %t2 = v128.int.insert i32x4 %lane %base %t1
\end{lstlisting}
The next variation of the SIMD vector load operation is the `load-lane' load operation. The example above demonstrate the procedure with a sample of WebAssembly's \texttt{v128.load32\_lane}  which reads a 32-bit integer from linear memory and inserts it into a specific lane of a given vector. SableWasm first lowers the load semantic using the same protocol as we discussed above and then inserts to the given vector using the \texttt{VectorInsert} instruction. Again, the WebAssmebly SIMD extension proposal defines `load-lane' load operation for all shapes of packed integers and floating-point numbers. In WebAssembly SIMD load operation variations, one may already notice that we only have width associate with them instead of types. This is because WebAssembly SIMD operations do not distinguish the shape for the vector. Hence, there is no difference in load a 32-bit integer and a single-precision floating number, as they both consume 32-bit storage. But in SableWasm, we distinguish between packed integers and packed floating-point numbers for SIMD instruction shape record.  On the other hand, SableWasm also erases shape information from the vector value, and it is the responsibility of the instruction to interpret the value correctly. Thus, when we perform load operation, we always assume that we are loading packed integers. In the examples above, the 32-bit load with translate to `load a 32-bit integer'.

\paragraph{SIMD load and extend} \quad
\begin{lstlisting}[basicstyle=\linespread{1}\small\ttfamily, language=SableWasmMIR, mathescape=true]
[..., %base i32] v128.load16x4_s offset=%offset align=%align $\Longrightarrow$
    %addr = int.add %base %offset
    memory.guard %mem %addr 8
    %t1 = load.64 v128 %addr 8
    %t2 = cast i32x4.extend.low.i16x8.s %t1
[..., %base i32] v128.load16x4_u offset=%offset align=%align $\Longrightarrow$
    %addr = int.add %base %offset
    memory.guard %mem %addr 8
    %t1 = load.64 v128 %addr 8
    %t2 = cast i32x4.extend.low.i16x8.u %t1
\end{lstlisting}
The last variation of load operation is the `load-and-extend' load operation. It is a combination of partial load and extension on the lower half of 128-bit vectors. In the example above we present examples for \texttt{v128.load16x4\_s} and \texttt{v128.load16x4\_u}. The previous instruction loads four 16-bit integers into lower lanes of the vector and performs signed-extension on the result to get a packed 32-bit integer vector. \texttt{v128.load16x4\_u} performs a similar operation, except that it performs zero-extension instead of signed-extension. A quick reminder, SableWasm MIR \texttt{Load} instruction can apply to any primitive value type and supports partial loading by annotating with a smaller load-width. In the case of the partial load, SableWasm MIR \texttt{Load} always loads bytes starting from the least significant bit and performs zero-extension on the result. SableWasm takes advantage of \texttt{Load} instruction's design when lowering the `load-and-extend' load operation. In the example above, we partially load a 128-bit vector with a 64-bit value which corresponds to loading four 16-bit integers from the linear memory. Note that this \texttt{Load} instruction yields a vector of 16-bit integers with four zero values in its higher lanes and loaded values in its lower lanes. Thus, we only need to perform a \texttt{Cast} operation with opcode \texttt{i32x4.extend.low.i16x8.s} to reach to the desired result. SableWasm treats \texttt{v128.load16x4\_u} using similar procedure, except that it use zero-extension instead of signed-extension. Finally, like other load operation variations discussed above, WebAssembly defines the `load-and-extend' load operation for all packed integer and packed floating-point numbers.

\paragraph{SIMD store lane} \quad
\begin{lstlisting}[basicstyle=\linespread{1}\small\ttfamily, language=SableWasmMIR, mathescape=true]
[..., %base i32, %val v128]
v128.store32_lane offset=%offset align=%align lane=%lane $\Longrightarrow$
    %addr = int.add %base %offset
    memory.guard %mem %addr 4
    %t1 = v128.int.extract i32x4 %val %lane
    store.32 %mem %addr %t1
\end{lstlisting}
Similar to `load-lane' load operation variation, the WebAssembly SIMD extension proposal also defines direct lane store instruction for 128-bit vectors. The above example demonstrates the reduced rules for these instructions. Let's take \texttt{v128.store32\_lane} as example. SableWasm MIR first calculates the address and setup memory boundary check use a similar protocol as we have seen above. Then, it extracts the lane value by using \texttt{VectorExtract} instruction and stores them into linear memory. Like WebAssembly load instructions, the store instruction does not distinguish between packed integers from packed floating-point numbers. In SableWasm, we always assume the store vector is packed integers.


\section{Analysis Framework}
\begin{figure}
    \centering
    \includegraphics[width=\textwidth]{Images/4.MIR/analysis-framework.pdf}
    \caption{SableWasm MIR Analysis and Optimization Framework}
    \label{fig:sablewasm-mir-analysis-framework}
\end{figure}
SableWasm also implements an analysis and optimization framework over middle-level intermediate representation (MIR). The framework consists of two parts, passes and drivers. SableWasm analysis and transformation framework only provides essential support for managing passes, compared to other more advanced frameworks, such as McSAF\cite{mcsaf}, an optimization framework for MATLAB language. Figure~\ref{fig:sablewasm-mir-analysis-framework} illustrates the current state of the framework in SableWasm. Currently, we implement three different drivers. \texttt{SimpleModulePassDirver} accepts module passes and operates on the module level. At the time of thesis writing, we haven't explored the inter-procedure analysis for SableWasm MIR in detail, and the only module pass implemented is the pretty print pass. In the future, one can add inter-procedure analysis to SableWasm, by implementing the \texttt{ModulePass} interface. The second driver is the \texttt{SimpleFunctionPassDriver}. As its name suggests, it manages \texttt{FunctionPass} instead. \texttt{FunctionPass} implements intra-procedural analysis that operates over basic blocks. SableWasm currently implements multiple intra-procedural analyses, such as dominator tree construction and local variable numbering. We will cover these passes in detail in this section. The last driver in SableWasm is \texttt{SimpleForEachFunctionPassDriver} which is a wrapper class for \texttt{SimpleFunctionPassDriver}. It works with \texttt{FunctionPass} but takes module as argument. It will apply the pass for each function within the module and terminates if they reach a fixpoint. McSAF manages the analysis result for the passes and automatically detects if the analysis reaches fixpoint. SableWasm analysis framework requires more manual work when designing passes and requires passes to report whether they have reached fixpoint to the driver correctly via pass's return value.

\input{Chapters/4.MIR/4.3.1.Dominators_and_Dependence}
\subsection{Control-Flow Graph Simplification}
`\subsection{Type Infer}

This section presents the type system for SableWasm MIR. SableWasm MIR is a statically typed language with a pretty straightforward type system. However, one may already notice that SableWasm MIR does not annotate every instruction with a type, unlike many other compiler intermediate representations. Instead, SableWasm computes the type for value on-demand via a set of type infer rules. The type system for SableWasm MIR generalizes from the MVP WebAssembly type system and its extension proposals with a few modifications. The formal definition for SableWasm MIR types are as follow,

\begin{lstlisting}[basicstyle=\linespread{1}\ttfamily, mathescape=true]

$\langle$primitive_type$\rangle$ ::= i32 | i64 | f32 | f64 | v128
$\langle$tuple_type$\rangle$     ::= (N, $\langle$primitive_type$\rangle$$\dots$)
$\langle$type$\rangle$           ::= $\langle$primitive_type$\rangle$ | <tuple_type$\rangle$ | () | $\bot$

\end{lstlisting}

Here we will skip the discussion for \emph{primitive type} and the type checking rules for its corresponding instructions as they are equivalent to the MVP WebAssembly type system. One should consult the specification for more details. The \emph{tuple type} consists of an unsigned integer and a list of primitive types. They model the return types of multi-value return functions or \texttt{Pack} instructions. Finally, we introduce the unit type, $()$ and bottom type, $\bot$. One can consider the unit type as \texttt{void} in the C programming language. They represent no value present, but the type is valid. On the other hand, the bottom type, $\bot$, signals that the pass can not assign any valid type to the term. In the rest of this section, we will focus on our discussion on extensions made to major WebAssembly extension proposals, multi-value and SIMD operation.

\paragraph{Multi-value} WebAssembly multi-value extensions allow functions to have more than one return values, which is quite interesting. Usually, low-level bytecode representation does not directly support this feature and usually only appears in higher-level language design, such as Python. In section 4.1.3, we introduced two instructions \texttt{Pack} and \texttt{Unpack}, along with how we represent multi-value for functions. As a quick recap, SableWasm uses tuple to denote the multi-value return for functions. \texttt{Pack} instruction collects values and constructs a tuple containing them, while on the other hand, \texttt{Unpack} extracts primitive values from tuples. Let's focus on the \texttt{Pack} instruction first. The typing rule for \texttt{Pack} is straightforward. If we can infer types for all candidate values, we say that the \texttt{Unpack} instruction has a tuple type consisting of the number of candidate values and a list of element types. On the other hand, if any of the candidate values result in a non-primitive type, the \texttt{Unpack} instruction is $\bot$ type. More formally,
$$
    \frac{\Gamma \vdash v_0 \Rightarrow t_0, \dots, v_n \Rightarrow t_n \qquad \forall i, t_i \in primitives}{\Gamma \vdash \text{\textbf{pack} } v_0, \dots, v_n \Rightarrow \langle n, t_0 \dots t_n \rangle}
    \qquad
    \frac{\Gamma \vdash \exists i, v_t \notin primitives}{\Gamma \vdash \text{\textbf{pack} } v_0, \dots, v_n \Rightarrow \bot}
$$
Here the set $primitives$ is the set of all possible primitive types in the SableWasm MIR type system. For \texttt{Unpack} instructions, the type checker will first check if the immediate index is within the tuple size. If the index is out of bounds, the type checker will assign the instruction with bottom type $\bot$. Otherwise, it will take the type from the tuple specified by the index. Formally,
$$
    \frac{\Gamma \vdash v \Rightarrow \langle n, t_0 \dots t_n \rangle \qquad k < n}{\Gamma \vdash \text{\textbf{unpack } k v} \Rightarrow t_k}
    \qquad
    \frac{\Gamma \vdash v \Rightarrow \langle n, t_0 \dots t_n \rangle \qquad k \geq n}{\Gamma \vdash \text{\textbf{unpack } k v} \Rightarrow \bot}
$$
We also generalize the function type in WebAssembly, and SableWasm MIR's function type will always have a single return value. We use the following strategy to map WebAssembly's function type into SableWasm MIR function type. In the case where there are no return values, we translate the return type into unit type. For example, SableWasm translate \texttt{[i32] -> []} into \texttt{[i32] -> ()}. On the other hand, if the function type has exactly one return value, the translation rule is trivial. Finally, when there are multiple return values, we pack them into a single tuple. For example, SableWasm use \texttt{[i32] -> (2, i32, f32)} to represent \texttt{[i32] -> [i32, f32]}  in WebAssembly.

\paragraph{SIMD operation}
Section 4.1.3 presents the instruction design in SableWasm MIR. We mentioned that WebAssembly's 128-bit vector value, added by the SIMD operation extension proposal, does not store their shape information in the type. Therefore, it is the instruction that is responsible for the instructions to interpret the shape correctly. WebAssembly's design gives us two choices in SableWasm when designing a type system for vector operations. First, we can follow the procedure in SableWasm, erase all the shape information for the value, and carefully plan the instruction semantics to make sure that all the operations have defined behaviour at runtime. Second, another approach is to add shape information back to the value. If there is a mismatch in shape formation, either the translation visitor can insert a bit cast, or the type checker can reject the program. In SableWasm MIR, we take the first approach by erasing all the shape information from the vector values. The later chapter on backend design will introduce the second approach in detail. The semantics for SIMD instruction in SableWasm MIR follows the WebAssembly's specification. We always store the value using the little-endian method, and the vectors start their first lane from the least significant bit.

In this section, we talk about the type infer pass in SableWasm MIR. Similar to the dominator analysis we seen in section 4.3.1, the type infer pass does not optimize the control-flow graph. But they are critical in the backend when we lower the SableWasm MIR into LLVM. We will come back to this in detail in chapter 5.
\subsection{Redundant Local Variable Elimination}

\begin{figure}[ht]
    \lstinputlisting[
        basicstyle=\linespread{1}\small\ttfamily,
        language=SableWasmMIR,numbers=left
    ]{Code/4.MIR/memcpy.mir}
    \caption{Redundant local variable elimination example}
    \label{fig:redundant-local-elem}
\end{figure}

In this section, we are focusing on another common problem that appeared in translated WebAssembly programs. From the WebAssembly validation rule, one may notice that, in MVP, there is no way for instructions in a \texttt{block}, \texttt{if}, or \texttt{loop} to access values beyond their scope. MVP WebAssembly adopts this rule to simplify the validation rules and ensure the safety of the generated module. However, this leads to poor performance in generated code. The WebAssembly compiler needs to push them to the local pool first, and later load them from the pool to push values into the nested expression within the structured control-flow constructs. A non-optimized runtime implementation might emit a store operation to stack memory and a load operation to stack memory. Figure~\ref{fig:redundant-local-elem} demonstrate this problem with an concrete example. The code is selected from a two-dimensional matrix multiplication benchmark test in C programming language and compiled to WebAssembly with WASI enabled Clang compiler \footnote{WASI SDK: \url{https://github.com/WebAssembly/wasi-sdk}}. Here we turn off the multivalue extension to give a better understanding of the problem. The function implements the memory copy procedure, \texttt{memcpy}, that appears in the standard C library. Here we only present a small snippet of the entire function as the original function contains more than 600 instructions and does not fit the thesis length. But, the problem appears in numerous locations in the entire generated control-flow graph. Line 13 and line 18 in figure~\ref{fig:redundant-local-elem} contain a pair of redundant local variable load and store instructions. So, why do we have these redundant load and store instructions here? One may notice that \texttt{\%BB:2} is the condition block that generated in the translation pattern for \texttt{if} instruction, and \texttt{\%BB:3} is the false block for that \texttt{if} instruction. As in MVP WebAssembly, the only possible type for \texttt{if} instruction is \texttt{[i32] -> []}, which consume exactly only one value as condition from the operand stack. Hence, the additional value needs to be awkwardly pass through via the local pool, in this case, value \texttt{\%23}. A better control flow graph should eliminate this \texttt{LocalGet} instruction in line 18 by replacing with \texttt{\%23}, and even better the \texttt{LocalSet} at line \texttt{\%13} if there are no value depends on this store operation.

To address these problems, we implement the redundant local elimination that derived from the point-to analysis\cite{alias-sable, point-to-microsoft, point-to-survey}. The accurate full-scale point-to analysis is hard. In fact, the problem is undecidable \cite{point-to-undecidable}. Compared to the full-scale point to analysis, the SableWasm MIR redundant local variable elimination is relatively simple. First, we only have one layer of indirection. In a full-scale alias analysis, one might need to construct a graph to get the whole picture. In SableWasm's redundant local elimination analysis, we only have one layer of indirection, a value referring to a local variable. Second, there are no complex data construct in WebAssembly. Finally, one complicated problem in alias analysis is around \texttt{Call} instructions. In a full-scale point-to analysis, one needs to utilize inter-procedural analysis to compute the result accurately. However, in SableWasm MIR, the local variable is private to its enclosing function, and the local alias can only happen within the function. Thus, when performing the local variable alias analysis, we can safely assume the calls do not affect the result. The redundant local variable elimination analysis splits into three parts, local variable aliasing, local get elimination and local set elimination.

\paragraph{Local variable aliasing}
Local variable aliasing analysis is the first part of the redundant local variable elimination analysis, and it is a forward data-flow analysis. It computes the relationship between local variables and values. We will first go through the result representations in the analysis. We record the possible alias using pairs $\langle \%local, \%value \rangle$, indicating that $\%local$ may contain value of $\%value$. Additionally, we introduce a special value $\%zero$, referring to the implicit zero-initializer at the beginning of the function. The initialization for `In' sets is straightforward. We initialize its `In' set with all locals implicit initialized to zero for the entry block and empty for any other basic blocks. Initially,  any basic block except the entry block has no information on local variable aliasing, and later in the analysis, we will incrementally collect them. The formal definition for `In' sets initialization is as follows. Here we use $locals$ to denote all local variables defined in the function, and $BB_{entry}$ refers to the entry block.
\begin{align*}
    IN(BB_{entry}) & = \{ \langle x, \%zero \rangle : x \in locals \} &                                 \\
    IN(BB)         & = \{ \}                                          & \forall BB . BB \neq BB_{entry}
\end{align*}
Next is the merge operator for the analysis. The merge operation for the local variable aliasing is a simple union operator, as the data flow may come from any predecessor blocks. The last part of the analysis is the `Gen' set and `Kill' set for instructions. In local aliasing analysis, we only need to consider for \texttt{LocalSet} instructions as they are the only instructions that create a point-to relationship between local variables and values. For clarity, let's assume the \texttt{LocalSet} instruction is \texttt{local.set \%local \%value} which set the local variable $local$ with $value$. The `Gen' set for the instruction is quite straightforward, and it is $\langle local, value \rangle$. The `Kill' set is more interesting. It is the set of pairs that has local variable $local$ because, after a \texttt{LocalSet} instruction, any point-to relationships for this local variable are outdated. More formally,
\begin{align*}
    \bowtie(BB)                                   & =
    \bigcup_{pred \in Pred(BB)} OUT(BB)               \\
    GEN(\text{\textbf{local.set} } local\ value)  & =
    \{ \langle local, value \rangle\}                 \\
    KILL(\text{\textbf{local.set} } local\ value) & =
    \{ \langle local, v \rangle : \forall v \in values \}
\end{align*}
Here we use $values$ to denote the set of all available values within the control-flow graph. The stop criteria for the analysis are simple. If there are no more changes in the result sets, the local variable aliasing analysis will terminate. It's easy to show that the process will eventually complete. The size of all results sets is non-decreasing between iterations, and there is an upper bound for the set cardinality. In this paragraph, we present the local variable aliasing analysis we used in SableWasm MIR. It is closely related to the existing point-to analysis and aliasing analysis.

\paragraph{Local variable get elimination}
After we compute the relationship between local variables and the values, we can then start performing local variable eliminations. Here, we begin by iterating through all instructions, and for \texttt{LocalGet} instructions, we exam the local variable aliasing results at that point. If the aliasing is unique, we can proceed to the next step of elimination; otherwise, we will leave the \texttt{LocalGet} unchanged. If the aliasing refers to $\%zero$, we will replace the \texttt{LocalGet} instructions with a \texttt{Constant} instructions with zero value. On the other hand, if the aliasing refers to a unique value, we will replace any occurrence of \texttt{LocalGet}'s value with it. Note that this replacement does not violate the SSA form of the control-flow graph. Assume there is an instruction that refers to the value yield by \texttt{LocalGet}, then its enclosing basic block is either dominated by the \texttt{LocalGet}'s block, or that instruction is a Phi node. One may show that the unique point-to value must dominate the \texttt{LocalGet}'s block; otherwise, the local variable refers to more than one value. Thus, if we replace a dominating value with another dominating value, the SSA constraints still hold.

\paragraph{Local variable set and local variable elimination}
The last part of the analysis is the local variable set and local variable elimination. Once we finished the local variable get elimination, we can reduce these problems into dead-code elimination. SableWasm MIR consider a \texttt{LocalSet} instruction dead if the local variables has no \texttt{LocalGet} refers to it. One may notice that this problem is isomorphic to liveness analysis. Indeed, in SableWasm MIR, we perform a liveness analysis on local variables to determine which local variables are still alive based on the \texttt{LocalGet} instructions and perform dead \texttt{LocalSet} elimination based on the result. We will skip the details on liveness analysis here as they are quite standard in compiler implementations. One could consult the paper\cite{fast-liveness} for more details. Finally, we can perform dead local variable elimination. Compare to dead \texttt{LocalSet} elimination, removing redundant local variable is quite straightforward. We only need to iterate through the use-site list to check if there exists any reference to the local variable; if not, we remove the instruction from the function.