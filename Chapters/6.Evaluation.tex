\chapter{Evaluation}

In the previous chapters, we presented the design of the SableWasm compiler and
runtime. This chapter will focus on the performance evaluation in terms of the
execution speed of the generated shared libraries. Here, we focus on three
research problems. First, how does SableWasm perform compared to other
WebAssembly runtime environment implementations? Second, does the optimization
over the input WebAssembly module affect the overall performance? Finally, how
much does the WebAssembly SIMD extension improve comparing to optimized scalar
counterparts? We will first present the setup for experiments used when
investigating three questions, and later, the experimental results for each one
of them.

\section{Experiment Setup}

This section presents the setup for the experiments in the remaining part of the
chapter. We conduct the benchmarks on the same server for all experiments.
The experiments were performed on a six-core Intel Core processor at a 3.7 GHz
standard clock frequency and with  an L3 cache of 12 MiB. Additionally, the
server runs Ubuntu 18.04 with Linux kernel version 4.15.0 and 32GiB of memory.
When measuring the performance, we execute each benchmark ten times in
succession to minimize the measurement error as some of the benchmarks take less
than a second to complete. Finally, the final benchmark result is the average
among ten runs except the highest and the lowest. For the benchmark subject,
we choose three different benchmark suits, the Polyhedral benchmark suite
(Polybench), the Ostrich benchmark suite (Ostrich), and the NAS parallel
benchmarks (NPB).

\begin{table}
    \centering
    \begin{tabular}{|l|l|}
    \hline
    \textbf{Benchmark Name} & \textbf{Description}                                \\ \hline
    2mm                     & 2 matrix multiplication (D = A.B; E = C.D)          \\ \hline
    3mm                     & 3 matrix multiplication (E = A.B; F = C.D; G = E.F) \\ \hline
    adi                     & alternating direction implicit solver               \\ \hline
    atax                    & matrix transpose followed by vector multiplication  \\ \hline
    bicg                    & BiCG sub kernel of BiCGStab linear solver           \\ \hline
    cholesky                & Cholesky decomposition                              \\ \hline
    correlation             & correlation computation                             \\ \hline
    covariance              & covariance computation                              \\ \hline
    doitgen                 & multiresolution analysis kernel (MADNESS)           \\ \hline
    durbin                  & Toeplitz system solver                              \\ \hline
    dynprog                 & dynamic programming (2D)                            \\ \hline
    fdtd-2d                 & 2D finite different time domain kernel              \\ \hline
    fdtd-apml               & FDTD using anisotropic perfectly matched layer      \\ \hline
    gauss-filter            & gaussian filter                                     \\ \hline
    gemm                    & matrix-multiply (C = alpha.A.B + beta.C)            \\ \hline
    gemver                  & vector multiplication and matrix addition           \\ \hline
    gesummv                 & scalar, vector and matrix multiplication            \\ \hline
    gramschmidt             & Gram-Schmidt decomposition                          \\ \hline
    jacobi-1D               & 1D Jacobi stencil computation                       \\ \hline
    jacobi-2D               & 2D Jacobi stencil computation                       \\ \hline
    lu                      & LU decomposition                                    \\ \hline
    ludcmp                  & LU decomposition (different implementation)         \\ \hline
    mvt                     & matrix vector product and transpose                 \\ \hline
    reg-detect              & 2D image processing                                 \\ \hline
    seidel                  & 2D Seidel stencil computation                       \\ \hline
    symm                    & symmetric matrix multiplication                     \\ \hline
    syr2k                   & symmetric rank-2k operations                        \\ \hline
    syrk                    & symmetric rank-k operations                         \\ \hline
    trisolv                 & triangular solver                                   \\ \hline
    trmm                    & triangular matrix multiplication                    \\ \hline
\end{tabular}
    \caption{the Polyhedral benchmark suite (Polybench)}
    \label{tbl:polybench}
\end{table}

\paragraph{Polybench}
The Polyhedral benchmark suite (Polybench) \cite{polybench} contains a group
of small math kernel functions as shown in table~\ref{tbl:polybench}. The
description table is adjusted from official Polybench documentation
\footnote{Polybench:
    \url{http://web.cse.ohio-state.edu/~pouchet.2/software/polybench/}}. In the
WebAssembly announcement paper \cite{10.1145/3062341.3062363}, the community
also chose Polybench as the evaluation subject. However, one problem is that
the Polybench is in C. Therefore, the researchers cross-compiled the benchmark
using a modified Clang compiler with an LLVM WebAssembly backend. However, there
is no standardized system interface, such as WASI, proposed by the community
when publishing the paper. Hence, the experiment is measured with an external
clock, and all features that require system interaction are disabled. On the
other hand, when evaluating SableWasm, we use a WASI-enabled
\footnote{WASI SDK: \url{https://github.com/WebAssembly/wasi-sdk}}
Clang compiler to cross-compile the WebAssembly modules into WebAssembly
modules. Each benchmark reports its execution time by issuing syscalls to the
runtime environment, which in theory, should yield more accurate results,
especially for a just-in-time (JIT) runtime environment.

\begin{table}
    \centering
    \begin{tabular}{|l|l|}
    \hline
    \textbf{Benchmark Name} & \textbf{Description}                                        \\ \hline
    back-prop               & backward propagation in a layered neural network            \\ \hline
    bfs                     & breadth-first search in a randomly generated graph          \\ \hline
    crc                     & CRC error-detecting algorithm                               \\ \hline
    fft                     & fast Fourier transform                                      \\ \hline
    hmm                     & forward-backward algorithm over a hidden Markov model       \\ \hline
    lavamd                  & 3D space particle simulation                                \\ \hline
    lud                     & LU decomposition                                            \\ \hline
    nqueens                 & N-queen problem solver                                      \\ \hline
    nw (needle)             & find optimal alignment of two protein sequences             \\ \hline
    page-rank               & page-rank algorithm to measure the importance of a web site \\ \hline
    spmv                    & sparse matrix multiplication with a vector                  \\ \hline
    srad                    & diffusion method for ultrasonic and radar imaging           \\ \hline
\end{tabular}
    \caption{the Ostrich benchmark suite (Ostrich)}
    \label{tbl:ostrich}
\end{table}

\paragraph{Ostrich}
The second benchmark suite we used was the Ostrich benchmark suite
\cite{ostrich}, illustrated in table~\ref{tbl:ostrich}. Comparing to the
Polybench, Ostrich focuses on larger scientific problems instead of
computation kernels. The Ostrich benchmark suite supports multiple programming
languages, such as Javascript and C. Here, we prepare the WebAssembly module
similar to the Polybench benchmark suite with a WASI-enabled Clang compiler.
However, unlike the Polybench benchmark suite, which does not require any
modification on the source code, we need to tweak the Ostrich benchmark code
due to the limitations of WebAssembly specification. This includes hard-coding
the command-line arguments and replacing throwing an exception with
calling the \texttt{exit} function.

\begin{table}
    \centering
    \begin{tabular}{|l|l|}
    \hline
    \textbf{Benchmark Name} & \textbf{Description}                                 \\ \hline
    IS                      & integer sort (bucket sort)                           \\ \hline
    EP                      & Marsaglia polar method for generating random numbers \\ \hline
    CG                      & estimate the smallest eigenvalue of a SPD matrix     \\ \hline
    MG                      & multi-grid on a sequence of meshes                   \\ \hline
    FT                      & fast Fourier transform                               \\ \hline
    BT                      & block tri-diagonal solver                            \\ \hline
    SP                      & scalar penta-diagonal solver                         \\ \hline
    LU                      & lower-upper solver                                   \\ \hline
\end{tabular}
    \caption{the NAS parallel benchmark suite (NPB)}
    \label{tbl:npb}
\end{table}

\paragraph{NPB}
The last benchmark suite we selected for evaluating SableWasm is the NAS
parallel benchmark suite \cite{npb}, shown in the table~\ref{tbl:npb}. We
choose this benchmark because of its parallel nature, as the third research
question focuses on the SIMD instruction operations. However, the original
NPB benchmark suite is in Fortran, and, at the time of thesis writing, there is
no cross-compiler from Fortran to WebAssembly. Hence, we choose an OpenMP
variant instead \footnote{NPB OpenMP C:
    \url{https://github.com/benchmark-subsetting/NPB3.0-omp-C}}. Although the
currently WASI-enabled Clang does not support OpenMP, we can still
cross-compile into WebAssembly, as OpenMP code trivially reduces to C.

This section presents the benchmark environment and test cases for the
experiments later in the chapter. One may notice some duplication among three
benchmark suites, such as the upper-lower matrix decomposition (LU, ludcmp)
and fast Fourier transform (FT, fft). However, we will still treat them as
different individual test cases for all of them, as they come with various
implementations and may lead to performance differences. Another problem that
arises when preparing WebAssembly modules for benchmark suits is that some of
the generated modules from the WASI-enabled Clang compiler have unexpected
behaviour. In NPB, although the WASI-enabled Clang compile can successfully
translate all test cases for all eight benchmark cases, there are two among
eight test cases that have different behaviour compared to their native
counterparts. For example, the WebAssembly module for the IS benchmark case
has a memory access out-of-bounds error for native and optimized translation.
Also, the module for EP failed when compiled with the optimization flag enabled
in the WASI-enabled Clang compiler. We suspect that some unknown bugs in the
toolchain may exist as it is still under active development. Another
possible cause for the problem is that the OpenMP implementation may contain
non-standard operations that result in undefined behaviour. We also test the
generated modules against several other WebAssembly runtime environments, and
the result is consistent. The last problem we encountered during benchmarking is
around WebAssembly SIMD operation extensions. As the extension is still under
standardization, most runtime environments only support a subset of all
instructions. Hence, when comparing SIMD operations, some of the benchmark
results are infeasible. However, we still manage to collect SIMD operation
performance data for most of the benchmark cases.

\section[RQ1: How does SableWasm perform compare to others?]{
  {\large RQ1: How does SableWasm perform compare to others?}}

This section will compare SableWasm performance against several other
WebAssembly runtime environments, specifically Wasmtime and Wasmer. We will
benchmark three implementations over naive (\texttt{-O0}),
optimized (\texttt{-O3}), and SIMD-enabled optimized (\texttt{-O3 -msimd128})
WebAssembly modules compiled from the source. One can consider Wasmtime
\footnote{Wasmtime: \url{https://github.com/bytecodealliance/wasmtime}} as
the `reference' implementation of WebAssembly out of the browser and it is
maintained by the WebAssembly community group. The system is built upon the
custom compile framework, Cranelift \footnote{Cranelift:
    \url{https://github.com/bytecodealliance/wasmtime/tree/main/cranelift}}.
Currently, both Cranelift and Wasmtime are still under active development and
subject to changes in the future. Here, in this project, we anchor our
Wasmtime at version 0.26.0. Wasmer \footnote{Wasmer: \url{https://wasmer.io/}}
is another community approach for running WebAssembly sandboxed applications
outside of the browser. It comes with a package manager, WAPM
\footnote{WAPM: \url{https://wapm.io/}}, that distributes applications in
WebAssembly binary format. Wasmer supports three compiler backends, LLVM,
Cranelift, and a single-pass code generator for fast compilation. In this
chapter, we will focus on the LLVM and Cranelift variants of Wasmer. Similar to
Wasmtime, Wasmer is also under active development at the time of thesis writing,
and we fix the version of Wasmer at 1.0.2. Unlike SableWasm, an
ahead-of-time (AOT) compiler for WebAssembly modules, Wasmtime and Wasmer are
both just-in-time (JIT). Thus, when measuring the benchmark's performance, we
need to isolate the error induced by the compiler, such as compilation-overhead
and warm-up time. To eliminate the compilation-overhead, we measure the
execution time with the internal timing code by issuing syscalls to the WASI
layer. Further, we adjust the benchmark size for Ostrich and NPB so that each
benchmark case takes more than 10 seconds to compute to reduce the error
introduced by the JIT warm-up process.

Figures~\ref{fig:rq1-wasmtime-naive} (page~\pageref{fig:rq1-wasmtime-naive}) to
\ref{fig:rq1-wasmer-llvm-simd} (page~\pageref{fig:rq1-wasmer-llvm-simd}) present
the benchmark results. We normalize the data with respect to the SableWasm's
execution time and present them as speedups. A number higher than one means
that the SableWasm's performance is better than the candidate, and on the
other hand, a less than one speed-up refers to slow-down. The error bar is
calculated based on the 10th percentile and 90th percentile accordingly.

\begin{figure}
    \centering
    \begin{subfigure}[t]{\textwidth}
        \includegraphics[width=\textwidth]
        {Images/6.1.RQ1/polybench-wasmtime-naive.pdf}
        \caption{Polybench}
    \end{subfigure}
    \begin{subfigure}[t]{.45\textwidth}
        \includegraphics[width=\textwidth]
        {Images/6.1.RQ1/ostrich-wasmtime-naive.pdf}
        \caption{Ostrich}
    \end{subfigure}
    \begin{subfigure}[t]{.45\textwidth}
        \includegraphics[width=\textwidth]
        {Images/6.1.RQ1/npb-wasmtime-naive.pdf}
        \caption{NPB}
    \end{subfigure}
    \caption{Benchmarks with naive (\texttt{-O0}) on Wasmtime}
    \label{fig:rq1-wasmtime-naive}
\end{figure}

\begin{figure}
    \centering
    \begin{subfigure}[t]{\textwidth}
        \includegraphics[width=\textwidth]
        {Images/6.1.RQ1/polybench-wasmer-cranelift-naive.pdf}
        \caption{Polybench}
    \end{subfigure}
    \begin{subfigure}[t]{.45\textwidth}
        \includegraphics[width=\textwidth]
        {Images/6.1.RQ1/ostrich-wasmer-cranelift-naive.pdf}
        \caption{Ostrich}
    \end{subfigure}
    \begin{subfigure}[t]{.45\textwidth}
        \includegraphics[width=\textwidth]
        {Images/6.1.RQ1/npb-wasmer-cranelift-naive.pdf}
        \caption{NPB}
    \end{subfigure}
    \caption{Benchmarks with naive (\texttt{-O0}) on Wasmer(Cranelift)}
    \label{fig:rq1-wasmer-cranelift-naive}
\end{figure}

\begin{figure}
    \centering
    \begin{subfigure}[t]{\textwidth}
        \includegraphics[width=\textwidth]
        {Images/6.1.RQ1/polybench-wasmer-llvm-naive.pdf}
        \caption{Polybench}
    \end{subfigure}
    \begin{subfigure}[t]{.45\textwidth}
        \includegraphics[width=\textwidth]
        {Images/6.1.RQ1/ostrich-wasmer-llvm-naive.pdf}
        \caption{Ostrich}
    \end{subfigure}
    \begin{subfigure}[t]{.45\textwidth}
        \includegraphics[width=\textwidth]
        {Images/6.1.RQ1/npb-wasmer-llvm-naive.pdf}
        \caption{NPB}
    \end{subfigure}
    \caption{Benchmarks with naive (\texttt{-O0}) on Wasmer(LLVM)}
    \label{fig:rq1-wasmer-llvm-naive}
\end{figure}

For naive translated WebAssembly modules, shown in
figures~\ref{fig:rq1-wasmtime-naive} to \ref{fig:rq1-wasmer-llvm-naive},
SableWasm performs better than Wasmtime in most benchmark cases except seven
of them. We suspect that the slow-down comes from the excessive linear
memory access. In the current version of the WASI-enabled Clang compiler, a
naive translated module will use linear memory to simulate stack frame functions
instead of using local variables. This means that when writing to a function's
local variable, SableWasm needs to first load the linear memory base pointer
from the instance object, calculate the address, and then perform the memory
access. Making the case worse, the current SableWasm will always load the base
memory pointer even if a local variable already holds the base pointer. LLVM
cannot effectively eliminate these load instructions, as the linear memory base
pointer in the instance object is volatile. One possible solution to ease the
problem is to carefully annotate the instance object pointer so that the alias
analysis in LLVM can correctly identify these redundant load instructions.
On the other hand, SableWasm performs better than Wasmer with both Cranelift
or LLVM backend. This is quite interesting as the current SableWasm
is also built upon LLVM. We suspect that two factors are contributing to the
speedup. First, SableWasm employs several optimization techniques to improve
the quality of the generated LLVM intermediate representation. When designing
the translation patterns for lowering SableWasm MIR into LLVM IR, we notice
that the quality of LLVM IR has a significant impact on the result performance,
especially for auto-vectorization. Second, Wasmer supports many other safety
features that are not specified in the WebAssembly specification, such as
stack probing. These safety features impose performance drawbacks which may
also contribute to the performance difference.

\begin{figure}
    \centering
    \begin{subfigure}[t]{\textwidth}
        \includegraphics[width=\textwidth]
        {Images/6.1.RQ1/polybench-wasmtime-opt.pdf}
        \caption{Polybench}
        \label{fig:rq1-wasmtime-opt-polybench}
    \end{subfigure}
    \begin{subfigure}[t]{.45\textwidth}
        \includegraphics[width=\textwidth]
        {Images/6.1.RQ1/ostrich-wasmtime-opt.pdf}
        \caption{Ostrich}
    \end{subfigure}
    \begin{subfigure}[t]{.45\textwidth}
        \includegraphics[width=\textwidth]
        {Images/6.1.RQ1/npb-wasmtime-opt.pdf}
        \caption{NPB}
    \end{subfigure}
    \caption{Benchmarks with optimized (\texttt{-O3}) on Wasmtime}
    \label{fig:rq1-wasmtime-opt}
\end{figure}

\begin{figure}
    \centering
    \begin{subfigure}[t]{\textwidth}
        \includegraphics[width=\textwidth]
        {Images/6.1.RQ1/polybench-wasmer-cranelift-opt.pdf}
        \caption{Polybench}
    \end{subfigure}
    \begin{subfigure}[t]{.45\textwidth}
        \includegraphics[width=\textwidth]
        {Images/6.1.RQ1/ostrich-wasmer-cranelift-opt.pdf}
        \caption{Ostrich}
    \end{subfigure}
    \begin{subfigure}[t]{.45\textwidth}
        \includegraphics[width=\textwidth]
        {Images/6.1.RQ1/npb-wasmer-cranelift-opt.pdf}
        \caption{NPB}
    \end{subfigure}
    \caption{Benchmarks with optimized (\texttt{-O3}) on Wasmer(Cranelift)}
    \label{fig:rq1-wamer-cranelift-opt}
\end{figure}

\begin{figure}
    \centering
    \begin{subfigure}[t]{\textwidth}
        \includegraphics[width=\textwidth]
        {Images/6.1.RQ1/polybench-wasmer-llvm-opt.pdf}
        \caption{Polybench}
    \end{subfigure}
    \begin{subfigure}[t]{.45\textwidth}
        \includegraphics[width=\textwidth]
        {Images/6.1.RQ1/ostrich-wasmer-llvm-opt.pdf}
        \caption{Ostrich}
    \end{subfigure}
    \begin{subfigure}[t]{.45\textwidth}
        \includegraphics[width=\textwidth]
        {Images/6.1.RQ1/npb-wasmer-llvm-opt.pdf}
        \caption{NPB}
    \end{subfigure}
    \caption{Benchmarks with optimized (\texttt{-O3}) on Wasmer(LLVM)}
    \label{fig:rq1-wasmer-llvm-opt}
\end{figure}

\begin{figure}
    \centering
    \begin{subfigure}[t]{\textwidth}
        \includegraphics[width=\textwidth]
        {Images/6.1.RQ1/polybench-wasmtime-simd.pdf}
        \caption{Polybench}
    \end{subfigure}
    \begin{subfigure}[t]{.45\textwidth}
        \includegraphics[width=\textwidth]
        {Images/6.1.RQ1/ostrich-wasmtime-simd.pdf}
        \caption{Ostrich}
    \end{subfigure}
    \begin{subfigure}[t]{.45\textwidth}
        \includegraphics[width=\textwidth]
        {Images/6.1.RQ1/npb-wasmtime-simd.pdf}
        \caption{NPB}
    \end{subfigure}
    \caption{Benchmarks with SIMD extension (\texttt{-O3 -msimd128})
        on Wasmtime}
    \label{fig:rq1-wasmtime-simd}
\end{figure}

\begin{figure}
    \centering
    \begin{subfigure}[t]{\textwidth}
        \includegraphics[width=\textwidth]
        {Images/6.1.RQ1/polybench-wasmer-cranelift-simd.pdf}
        \caption{Polybench}
    \end{subfigure}
    \begin{subfigure}[t]{.45\textwidth}
        \includegraphics[width=\textwidth]
        {Images/6.1.RQ1/ostrich-wasmer-cranelift-simd.pdf}
        \caption{Ostrich}
    \end{subfigure}
    \begin{subfigure}[t]{.45\textwidth}
        \includegraphics[width=\textwidth]
        {Images/6.1.RQ1/npb-wasmer-cranelift-simd.pdf}
        \caption{NPB}
    \end{subfigure}
    \caption{Benchmarks with SIMD extension (\texttt{-O3 -msimd128})
        on Wasmer(Cranelift)}
    \label{fig:rq1-wasmer-cranelift-simd}
\end{figure}

\begin{figure}
    \centering
    \begin{subfigure}[t]{\textwidth}
        \includegraphics[width=\textwidth]
        {Images/6.1.RQ1/polybench-wasmer-llvm-simd.pdf}
        \caption{Polybench}
    \end{subfigure}
    \begin{subfigure}[t]{.45\textwidth}
        \includegraphics[width=\textwidth]
        {Images/6.1.RQ1/ostrich-wasmer-llvm-simd.pdf}
        \caption{Ostrich}
    \end{subfigure}
    \begin{subfigure}[t]{.45\textwidth}
        \includegraphics[width=\textwidth]
        {Images/6.1.RQ1/npb-wasmer-llvm-simd.pdf}
        \caption{NPB}
    \end{subfigure}
    \caption{Benchmarks with SIMD extension (\texttt{-O3 -msimd128}) on Wasmer(LLVM)}
    \label{fig:rq1-wasmer-llvm-simd}
\end{figure}

For optimized and SIMD-enabled input WebAssembly modules,
show in figures~\ref{fig:rq1-wasmtime-opt} (page~\pageref{fig:rq1-wasmtime-opt})
to \ref{fig:rq1-wasmer-llvm-simd} (page~\pageref{fig:rq1-wasmer-llvm-simd})
SableWasm performs on par with Wasmtime, except on benchmark case
\texttt{durbin}. One may also notice that the error for \texttt{durbin} in
figure~\ref{fig:rq1-wasmtime-opt-polybench} is more significant compared to
others. This is due to the nature of the \texttt{durbin} benchmark case.
The \texttt{durbin} benchmark contains a tiny computation kernel function and
only takes a few milliseconds to complete. For Ostrich and NPB, we can adjust
the benchmark size to reduce the measurement errors. However, this is not the
case for Polybench, as the input size is hardcoded. On the other hand,
SableWasm performs better than Wasmer in most of the benchmark cases. Here we
will take \texttt{floyd-warshall} as an example. The core computation
function in \texttt{floyd-warshall} is a nested for-loop that iteratively
multiplies then adds matrices. This operation is highly parallel. We notice
that the performance of SableWasm is approximately four times better in
optimized input and two times better for SIMD-enabled. Currently, there
seems no way to retrieve generated LLVM IR from Wasmer, and we can only
speculate on reasons based on the experiment results. We suspect that the
auto-vectorization may cause this in the LLVM framework. The four times and two
times speedup appears to align with the SIMD vector operations for packed
double-precision floating-point numbers. Thus, Wasmer may contain an awkwardly
generated LLVM intermediate representation that stops the auto-vectorization
pass to turn scalar code into the vectorized form.

\begin{table}
    \centering
    % Please add the following required packages to your document preamble:
% \usepackage{multirow}
% \usepackage[table,xcdraw]{xcolor}
% If you use beamer only pass "xcolor=table" option, i.e. \documentclass[xcolor=table]{beamer}

\begin{tabular}{ll|ccc}
    \multicolumn{2}{c|}{\textbf{Benchmark name}}                 & \textbf{Wasmtime}                   & \textbf{Wasmer (Cranelift)} & \textbf{Wasmer (LLVM)}                                    \\ \hline
    \rowcolor[HTML]{C0C0C0}
    \cellcolor[HTML]{C0C0C0}                                     & Naive                               & 1.16                        & 3.04                        & 1.46                        \\
    \rowcolor[HTML]{C0C0C0}
    \cellcolor[HTML]{C0C0C0}                                     & Optimized                           & 1.09                        & 1.77                        & 1.46                        \\
    \rowcolor[HTML]{C0C0C0}
    \multirow{-3}{*}{\cellcolor[HTML]{C0C0C0}\textbf{Polybench}} & SIMD-enabled                        & 1.09                        & 1.71                        & 1.47                        \\ \hline
                                                                 & Naive                               & 1.06                        & 2.39                        & 1.32                        \\
                                                                 & Optimized                           & 1.13                        & 1.83                        & 1.34                        \\
    \multirow{-3}{*}{\textbf{Ostrich}}                           & SIMD-enabled                        & 1.13                        & 1.79                        & 1.33                        \\ \hline
    \rowcolor[HTML]{C0C0C0}
    \cellcolor[HTML]{C0C0C0}                                     & {\color[HTML]{333333} Naive}        & {\color[HTML]{333333} 1.20} & {\color[HTML]{333333} 3.05} & {\color[HTML]{333333} 1.79} \\
    \rowcolor[HTML]{C0C0C0}
    \cellcolor[HTML]{C0C0C0}                                     & {\color[HTML]{333333} Optimized}    & {\color[HTML]{333333} 1.14} & {\color[HTML]{333333} 1.93} & {\color[HTML]{333333} 1.45} \\
    \rowcolor[HTML]{C0C0C0}
    \multirow{-3}{*}{\cellcolor[HTML]{C0C0C0}\textbf{NPB}}       & {\color[HTML]{333333} SIMD-enabled} & {\color[HTML]{333333} 1.15} & {\color[HTML]{333333} 1.85} & {\color[HTML]{333333} 1.69}
\end{tabular}

    \caption{Geometric mean of speedups compare to Wasmtime and Wasmer}
    \label{tbl:rq1-average}
\end{table}

In general, we conclude that SableWasm performs on par with Wasmtime on
optimized and SIMD-enabled WebAssembly modules and better than Wasmtime and
Wasmer for other benchmark cases, as shown in table~\ref{tbl:rq1-average}.

\section[RQ2: Does optimization over input modules matter?]{
  {\large RQ2: Does optimization over input modules matter?}}

The second problem we would like to investigate in the thesis is whether the
optimization over input WebAssembly modules affects their performance in
SableWasm. In theory, a perfect runtime system should recover all the missed
possible optimization from a naive translated WebAssembly module, which we
have seen in many other systems with two-phase compilation. The most well-known
example is perhaps Java. The first compiler, \texttt{javac}, translates Java
source files into bytecodes, and on the other hand, the Java virtual machine
(JVM) generates naive executable binaries based on the bytecodes at runtime.
\texttt{javac} will translate the source files faithfully, without complex
transformation and optimization, while the JVM optimizes the bytecodes
effectively. This design helps the system to achieve fast compilation while
ensuring the quality of the final generated code. In the current SableWasm,
however, the optimization over input WebAssembly modules does have a significant
impact on the overall performance. When investigating the problem, we compare
the SableWasm execution time under naive translated WebAssembly modules against
their optimized counterparts.

\begin{figure}
    \centering
    \begin{subfigure}[t]{\textwidth}
        \includegraphics[width=\textwidth]{Images/6.2.RQ2/polybench-opt-speedup}
        \caption{Polybench}
    \end{subfigure}
    \begin{subfigure}[t]{.45\textwidth}
        \includegraphics[width=\textwidth]{Images/6.2.RQ2/ostrich-opt-speedup}
        \caption{Ostrich}
    \end{subfigure}
    \begin{subfigure}[t]{.45\textwidth}
        \includegraphics[width=\textwidth]{Images/6.2.RQ2/npb-opt-speedup}
        \caption{NPB}
    \end{subfigure}
    \caption{Comparision between optimized and naive input modules}
    \label{fig:rq2}
\end{figure}

For most of the benchmarks, except \texttt{gramschmidt}, we have seen a
significant performance increase for optimized WebAssembly modules, as shown
in figure~\ref{fig:rq2}. Here we will take \texttt{trisolv} as an example. We
found many optimizations missing when comparing the LLVM intermediate
representation generated from a naive WebAssembly module against an optimized
one. The most notable difference is perhaps function inlining and loop
auto-vectorization. In LLVM generated by naive WebAssembly module, SableWasm
does not inline the primary function \texttt{kernel\_trisolv} into the main
function. Thus, matrices are passed as arguments to the computation kernel via
pointers, which holds back SableWasm from transforming the internal nested
loops into parallel form due to possible data dependencies. We suspect that the
translation patterns used in SableWasm when lowering WebAssembly bytecode into
SableWasm MIR confuse the LLVM backend. Another interesting question is why
\texttt{gramschmidt} has similar performance on both input WebAssembly modules.
When we compare the LLVM intermediate representation for both inputs, we found
that SableWasm can recover nearly all the optimization in the computation
kernel: the computation kernel for \texttt{gramschmidt} is extremely simple,
only consisting of three non-nested for loops. This further confirms our theory
on performance difference.

\section[RQ3: How much does SIMD extension improve in performance?]{
  {\large RQ3: How much does SIMD extension improve in performance?}}

\begin{figure}[h]
    \lstinputlisting[
        language=C,
        basicstyle=\linespread{0.5}\ttfamily\small,
        numbers=left
    ]{Code/6.Evaluation/gemm.kernel.c}
    \caption{Polybench \texttt{gemm} benchmark kernel}
    \label{fig:polybench-gemm-kernel}
\end{figure}

\begin{figure}
    \centering
    \begin{subfigure}[t]{\textwidth}
        \includegraphics[
            width=\textwidth
        ]{Images/6.3.RQ3/polybench-simd-speedup}
        \caption{Polybench}
    \end{subfigure}
    \begin{subfigure}[t]{.45\textwidth}
        \includegraphics[width=\textwidth]{Images/6.3.RQ3/ostrich-simd-speedup}
        \caption{Ostrich}
    \end{subfigure}
    \begin{subfigure}[t]{.45\textwidth}
        \includegraphics[width=\textwidth]{Images/6.3.RQ3/npb-simd-speedup}
        \caption{NPB}
    \end{subfigure}
    \caption{Comparision between SIMD-enabled and optimized input modules}
    \label{fig:rq3}
\end{figure}

\begin{figure}
    \centering
    \begin{subfigure}{\textwidth}
        \lstinputlisting[
            basicstyle=\linespread{0.5}\ttfamily\small,
            language=LLVM,
            numbers=left
        ]{Code/6.Evaluation/gemm.opt.ll}
        \caption{Optimized}
        \label{fig:polybench-gemm-opt-code}
    \end{subfigure}
    \begin{subfigure}{\textwidth}
        \lstinputlisting[
            basicstyle=\linespread{0.5}\ttfamily\small,
            language=LLVM,
            numbers=left
        ]{Code/6.Evaluation/gemm.simd.ll}
        \caption{SIMD-enabled}
        \label{fig:polybench-gemm-simd-code}
    \end{subfigure}
    \caption{Polybench \texttt{gemm} code snippet}
\end{figure}

The last question we would like to investigate in the thesis is how much
the WebAssembly SIMD operation extension improves performance compared to MVP
WebAssembly. In this experiment, we compare the execution time between
optimized WebAssembly modules and SIMD-enabled WebAssembly modules.
Figure~\ref{fig:rq3} illustrates the experiment results. For most benchmark
cases, the SIMD extension does not significantly improve the performance,
except for five cases in Polybench. Here we will take \texttt{gemm} as an
xample. Figure~\ref{fig:polybench-gemm-kernel} illustrates the computation
kernel of the benchmark, and it consists of a single nested loop that performs
floating-point mathematics over two matrices. When translating the optimized
(\texttt{-O3}) input WebAssembly module, illustrated in
figure~\ref{fig:polybench-gemm-opt-code}, SableWasm correctly performs loop
unrolling on the inner for-loop. However, the LLVM auto-vectorizer failed to
transform the scalar code into a parallel form. On the other hand, in the case
of SIMD-enabled WebAssembly module input (\texttt{-O3 -msimd128}), SableWasm
takes the direct hint from the WASI-enabled clang compiler and correctly emits
the 128-bit wide vector operations, as shown
figure~\ref{fig:polybench-gemm-simd-code}.

\section*{Conclusion}

In this chapter, we evaluated the performance of SableWasm in terms of three
aspects. First, we compared SableWasm's performance against several other
well-known WebAssembly runtime environments, Wasmtime and Wasmer. We concluded
that SableWasm performs on par with Wasmtime when the input module is optimized
and better in all other cases. Second, we investigated whether the optimization
over input modules affects the system's overall performance. Currently,
SableWasm heavily relies on the frontend compiler to emit efficient code, as
the performance gap between naive and optimized input modules is still quite
significant. Finally, we evaluated the effectiveness of WebAssembly SIMD
operation extensions. Our experiments identified several benchmark cases where
explicit SIMD instructions make a measurable difference in the execution speed.