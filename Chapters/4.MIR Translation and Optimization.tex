\chapter{Middle-level Intermediate Representation Translation and Optimization}
\label{chapter:mir-translation-optimization}

The previous chapter presented the SableWasm middle-level intermediate
representation (MIR), a static-single-assignment (SSA) control flow graph (CFG)
representation of a WebAssembly program. This chapter focuses on the translation
strategy used when lowering WebAssembly into the SableWasm MIR. We will first
start by presenting the translation patterns used and then discuss the analysis
and optimization framework.

\section{Translating WebAssembly to MIR}
\label{section:mir-translation}

In this section, we will cover the translation between WebAssembly bytecode and
SableWasm MIR. We have covered the design of SableWasm MIR instructions
previously. One may notice that for most of the instructions, especially for the
numerical operations, SableWasm MIR shares the same semantics as WebAssembly.
Hence, the translation rules for these instructions are pretty trivial, and we
will not cover them in detail in this section. Instead, this section will focus
on the translation rules for the structured control flow constructs and
WebAssembly instructions that require reduction during translation.

\subsection{Structured-Control-Flow Construct}

Translating from stack-based IR to register-based IR is not trivial, especially
when non-linear control flow structures appeared. This problem appeared in many
runtime system implementations, such as Numba \cite{numba}, a just-in-time (JIT)
compiler for Python. Usually, one needs some algorithm to recover the control
flow structure from annoying jump instructions. Luckily, in WebAssembly, we can
translate the stack-based bytecode into register-based basic blocks in linear
time, thanks to the structured-control-flow constructs and their validation
rules defined in WebAssembly. In this section, we will cover the translation
pattern used for WebAssembly's structured-control-flow constructs, namely
\texttt{block}, \texttt{if} and \texttt{loop}.

\begin{figure}
  \centering
  \includegraphics[width=\textwidth]{Images/4.MIR/translate-block.pdf}
  \caption{WebAssembly \texttt{block} translation pattern}
  \label{fig:translate-block}
\end{figure}

\begin{figure}
  \centering
  \includegraphics[width=\textwidth]{Images/4.MIR/translate-if.pdf}
  \caption{WebAssembly \texttt{if} translation pattern}
  \label{fig:translate-if}
\end{figure}

\paragraph{Block}
In the background chapter, we provide a general illustration of the three
structured-control-flow constructs. As a quick recap, \texttt{block} is the
simplest form of a structured-control-flow construct. It implicitly introduces
a label at the end of its enclosing instructions. A branching instruction
referring to this label will redirect the control flow to the end of the block.
Figure~\ref{fig:translate-block} illustrates the translation pattern for
WebAssembly \texttt{block} in SableWasm MIR. We will first clarify some of the
terminologies we used in the figure, and we will use the same terms later in
the \texttt{loop} and \texttt{if} pattern discussion. \emph{Expr Insert Point}
refer to the starting position for the generated instructions when we
recursively translate the instructions within the enclosing expression of the
\texttt{block} instruction. Furthermore, \emph{Label Insert Point} refer to the
position for generated instruction when we finish the recursive translation and
resume to the parent expression of the \texttt{block} instruction. A \emph{label
  stack entry} is a tuple consisting of a pointer to the landing BB, a list of
$\phi$ nodes expecting merge values, and a pointer to the \emph{label insert
  point}. The translation pattern for \texttt{block} is pretty simple; we
continue on the current BB and prepare the landing BB for the block instruction
as a branch instructions within the expression may refer to the label.
Additionally, to fully support multi-value extension in WebAssembly, we also
need to prepare the $\phi$ nodes in the landing BB. SableWasm generates the
$\phi$ nodes based on the type of the \texttt{block} instruction. WebAssembly
validation ensures that the expression within the \texttt{block} can access
exactly $m$ values from the stack and put $n$ values onto the stack. Finally,
we will append an unconditional branch to the landing BB because in WebAssembly,
if the control flow reaches the bottom of the \texttt{block} expressions, it
will implicitly fall through. For the operand stack, we will first pop $m$
values from the stack as \texttt{block} instruction's type suggests and push the
$\phi$ nodes as the result values. Now we need to set up the operand stack for
our the expression contained within the block. Again, due to the WebAssembly
validation rule, we need to insert a boundary before continuing.
Figure~\ref{fig:translate-block} represents this with the bold line in the
result operand value stack.


\paragraph{If}
The next control-flow structure defined WebAssembly is \texttt{if}.
WebAssembly's \texttt{if} is an expression instead of a statement that appears
in many other languages such as C. The \texttt{if} expression can yield some
values indicated by its type. Figure~\ref{fig:translate-if} illustrates the
translation patterns in SableWasm. There are two types of \texttt{if}
instruction defined in WebAssembly specification. The first case is a `partial'
\texttt{if} instruction, where it only contains the `true' branch. From
WebAssembly validation rules, it's easy to show that the only possible type is
\texttt{[i32]->[]}, even with the multi-value extension proposal. This implies
that the expression within the \texttt{if} instruction must start with an empty
operand stack. Hence, the translation pattern for the partial \texttt{if} is
quite straightforward: we only need to pop the condition value from the operand
stack and construct a conditional branch based on this value in the current BB.
On the other hand, we also have `full' \texttt{if} instructions with both `true'
expression and `false' expression. The validation rules ensure that both
expressions must have the same type. The translation pattern is more complex
compare to that of a `partial' \texttt{if}. In this case, we have to prepare the
landing BB similarly to what we did for the \texttt{block} construct. We need to
generate $n$ $\phi$ nodes for data-flow mergers from the true branch, the false
branch, and any possible branching instruction within both nested expressions.
Similarly, we need to pop $m$ values from the stack for operand values stack and
then push $n$ $\phi$ nodes. And, within both nested expressions, push $m$ values
back to the stack.

\begin{figure}
  \centering
  \includegraphics[width=\textwidth]{Images/4.MIR/translate-loop.pdf}
  \caption{WebAssembly \texttt{loop} translation pattern}
  \label{fig:translate-loop}
\end{figure}

\paragraph{Loop}
The last control-flow structure defined in WebAssembly is \texttt{loop}.
Figure~\ref{fig:translate-loop} gives a general illustration of SableWasm's
translation pattern for \texttt{loop} instructions. Similar to the `partial'
\texttt{if} we discussed in the previous paragraph, one can show that, under
WebAssembly's validation rules, the parameter types for the \texttt{loop}
instruction must equal to the result types. The \texttt{loop} instruction is
similar to the \texttt{block} instruction, except that if any branching
instruction refers to it, the branching instruction should transfer the control
flow to the start of the expression within the instruction instead of the end.
Thus, we need to prepare a standalone basic block for the nested expression in
\texttt{loop}, along with the $\phi$ nodes to merge value on each loop
iteration. Note that we also introduce $\phi$ nodes in the landing BB. One may
argue that there is no need for these $\phi$ nodes, as only one block can reach
the loop exit and no value merging will occur. Indeed, these $\phi$ nodes will
always be trivial $\phi$ nodes, which have only one possible value inflow.
However, this is due to the limitation of our translation framework.

In this section, we discussed the translation patterns for WebAssembly
structured control-flow constructs. Thanks to WebAssembly validation rules, the
types for these structured control-flow instructions explicitly mark value
merging and imply possible $\phi$ nodes. Furthermore, one can show that the
control graph generated above is indeed in SSA form. However, the directly
generated control flow graph is not easily understandable by users. This mainly
comes from two facts. First, the WebAssembly-targeting compiler may generate
awkward patterns to fit in the structured control-flow constructs. Second,
SableWasm translation patterns for structured-control flow constructs are not
optimal.
\subsection{Instruction Reduction}

This section will cover the instruction reduction rules used when lowering
WebAssembly bytecode to SableWasm MIR. In the background chapter, we mentioned
that one of WebAssembly's design goals is to be as compact as possible. Thus,
when the community designed the WebAssembly instruction set, they fused several
typical instruction sequences into single instructions. For example, SIMD vector
operation extension defines \texttt{v128.load8x8\_s} which first load 8
8-bit integers into a vector, and then sign-extends them into 16-bit
integers. Another example will be \texttt{v128.load32\_lane} which loads a
32-bit value, either a 32-bit integer or a single-precision floating-point
number into a given vector. Such design is understandable for WebAssembly as
binary size does matter when shipping applications over the internet. But, for
SableWasm, a static compiler, we focus more on the size of the instruction set
instead of the size of the intermediate representation. It is harder to write
analysis for a bloated instruction set, as one needs to consider more
instruction cases. Hence, when lowering WebAssembly bytecode to SableWasm MIR,
we replace some WebAssembly instructions with SableWasm MIR instructions
sequences.

\paragraph{Eqz} \quad
\begin{lstlisting}[
    basicstyle=\linespread{0.7}\small\ttfamily, 
    language=SableWasmMIR, 
    mathescape=true]
[..., %n i32] i32.eqz $\Longrightarrow$ %t0 = i32.const 0; %t1 = int.eq %n %t0
[..., %n i64] i64.eqz $\Longrightarrow$ %t0 = i64.const 0; %t1 = int.eq %n %t0
\end{lstlisting}
WebAssembly defines a unary \texttt{eqz} operations for all integer values. As
the name suggests, \texttt{eqz} compares the operand value against zero and
yields one if true, zero otherwise. In SableWasm MIR, we group all comparison
instructions into the \texttt{Compare} class, and \texttt{eqz} does not fit into
the class as it is not a binary operation. Hence we rewrite the \texttt{eqz} as
\texttt{Compare} instruction with opcode as \texttt{Eq}.

\paragraph{Load} \quad
\begin{lstlisting}[
    basicstyle=\linespread{0.7}\small\ttfamily, 
    language=SableWasmMIR, 
    mathescape=true]
[..., %base i32] i32.load offset=%offset align=%align $\Longrightarrow$
    %addr = int.add %base %offset
    memory.guard %mem %addr 4
    %t0 = load.32 i32 %mem %addr
[..., %base i32] i32.load16_s offset=%offset align=%align $\Longrightarrow$
    %addr = int.add %base %offset
    memory.guard %mem %addr 2
    %t0 = load.16 i32 %mem %addr
    %t1 = cast i32.extend.16.s %t0
[..., %base i32] i32.load16_u offset=%offset align=%align $\Longrightarrow$
    %addr = int.add %base %offset
    memory.guard %mem %addr 2
    %t0 = load.16 i32 %mem %addr
\end{lstlisting}
In the SableWasm instruction design section, we introduced the \texttt{Load} and
\texttt{MemoryGuard} in SableWasm MIR. A quick recap, SableWasm MIR
\texttt{Load} instruction, compare to its WebAssembly counterpart, assumes
access is in-bound, does not support offset attribute, and always performs
zero-extension on partial loads. Hence, to properly support WebAssembly's
\texttt{load} instructions, we need to reduce them with the strategy shown
above. For load instructions that do not require value extensions, such as
\texttt{i32.load}, we first calculate the actual starting address, perform a
memory boundary check with \texttt{MemoryGuard}, and then perform the memory
read. On the other hand, for a partial load operation, we need first to perform
the load operation using the same protocol as a normal load. Then, if a sign
extension is needed, we will add its corresponding cast instruction. In the
example above, we demonstrate this with WebAssembly's \texttt{i32.load16\_s}.
In this case, SableWasm appends a \texttt{Cast} instruction with opcode
\texttt{i32.extend.16} after the load operation.

\paragraph{Store} \quad
\begin{lstlisting}[
    basicstyle=\linespread{0.7}\small\ttfamily, 
    language=SableWasmMIR, 
    mathescape=true]
[..., %base i32, %val i64] i64.store offset=%offset align=%align $\Longrightarrow$
    %addr = int.add %base %offset
    memory.guard %mem %addr 8
    store.64 %mem %addr %val
[..., %base i32, %val i64] i64.store16 offset=%offset align=%align $\Longrightarrow$
    %addr = int.add %base %offset
    memory.guard %mem %addr 2
    store.16 %mem %addr %val
\end{lstlisting}
Similar to the \texttt{Load} instruction we discussed earlier, the
\texttt{Store} instruction also assumes the memory access is always in range and
does not provide the offset attribute. However, a \texttt{Store} instruction
will always perform truncation instead of extension. Further, the only possible
truncation is the bit-truncation by discarding bits starting from the most
significant bit. The instruction reduction rules for WebAssembly \texttt{store}
instructions is similar to those for \texttt{load} instructions. In the example
above, we demonstrate the rules with \texttt{i64.store} and its partial store
version, \texttt{i64.store16} which only stores the lowest two bytes into linear
memory. SableWasm inserts \texttt{MemoryGuard} instructions in a similar fashion
to \texttt{load} instructions. Note that we do not insert an explicit
\texttt{Cast} instruction to perform the truncation. A \texttt{Store}
instruction will implicitly truncate the value according to the store width; in
this case, it will truncate the 64-bit integer into a 16-bit integer.

\paragraph{SIMD extension proposal reduction rules} \quad
\begin{lstlisting}[
    basicstyle=\linespread{0.7}\small\ttfamily, 
    language=SableWasmMIR, 
    mathescape=true]
[..., %lhs v128, %rhs v128] v128.andnot $\Longrightarrow$
    %t0 = v128.not %rhs 
    %t1 = v128.and %lhs %t0
[..., %lhs v128, %rhs v128] i16x8.extmul_low_i8x16_s $\Longrightarrow$
    %t0 = cast i16x8.extend.low.i8x16.s %lhs
    %t1 = cast i16x8.extend.low.i8x16.s %rhs
    %t2 = v128.int.mul i16x8 %t0 %t1
[..., %lhs v128, %rhs v128] i16x8.extmul_low_i8x16_u $\Longrightarrow$
    %t0 = cast i16x8.extend.low.i8x16.u %lhs
    %t1 = cast i16x8.extend.low.i8x16.u %rhs
    %t2 = v128.int.mul i16x8 %t0 %t1
\end{lstlisting}
The SIMD extension proposal introduces approximately 240 instructions into the
WebAssembly instruction set. However, not all of them are simple single
operation instructions. The SIMD extension proposal also follows WebAssembly's
design goal to ensure the compactness of the generated program. The proposal
suggests reduction rules for several SIMD operation instructions, and in
SableWasm, we take advantage of them to reduce the size of the instruction set.
The first applicable instruction is the \texttt{andnot} operation for vectors.
The \texttt{andnot} is equivalent to performing bitwise `not' on the
right-hand-side operand, and then a bitwise `and' operation between the
left-hand-side operand and the temporary result. SableWasm reduces
\texttt{andnot} into a \texttt{not} instruction followed by a \texttt{and}
instruction, as shown in the example above. The second group of reducible
instructions is the \texttt{ExtMul} instructions. The SIMD extension proposal
defines \texttt{ExtMul} for all packed integer vectors except packed 64-bit
integers. They are equivalent to first widening the vector using the appropriate
extension and then multiplying two operands. In the example above, we
demonstrate with \texttt{i16x8.extmul\_low\_i8x16\_s} which performs an
\texttt{ExtMul} operation for packed 8-bit integers. SableWasm implements this
instruction by first performing a sign extension on the lower half of the vector
and multiplying the temporary result as shown above. SableWasm also applies a
similar procedure to \texttt{i16x8.extmul\_low\_i8x16\_u}, except that it uses
a zero-extension in the \texttt{Cast} instruction instead of sign-extension.

\paragraph{SIMD load with zero-padding} \quad
\begin{lstlisting}[
    basicstyle=\linespread{0.7}\small\ttfamily, 
    language=SableWasmMIR, 
    mathescape=true]
[..., %base i32] v128.load32_zero offset=%offset align=%align $\Longrightarrow$
    %addr = int.add %base %offset
    memory.guard %mem %addr 4
    %t1 = load.32 i32 %mem %addr
    %t2 = const v128 0
    %t3 = v128.int.insert i32x4 0 %t2 %t1
\end{lstlisting}
The WebAssembly SIMD extension proposal also introduces many variations of load
operations. The first variation is the `zero-padding' load operation. The
`zero-padding' load is equivalent to loading a scalar from the linear memory and
then inserting it into a zero-initialized vector. We demonstrate this with the
example above. We first use the protocol we discussed above to load a scalar
32-bit integer. Then, we insert it into a zero vector using
\texttt{VectorInsert} instruction. The WebAssembly SIMD extension proposal
defines `zero-padding' load operations for all packed integers and
packed float-point numbers. The reduction rules for them are similar to the
pattern above.

\paragraph{SIMD load and splat} \quad
\begin{lstlisting}[
    basicstyle=\linespread{0.7}\small\ttfamily, 
    language=SableWasmMIR, 
    mathescape=true]
[..., %base i32] v128.load32_splat offset=%offset align=%align $\Longrightarrow$
    %addr = int.add %base %offset
    memory.guard %mem %addr 4
    %t1 = load.32 i32 %mem %addr
    %t2 = v128.int.splat i32x4 0 %t1
\end{lstlisting}
The second variation of SIMD vector load is the `load-and-splat' load operation.
This type of load operation is a combination of scalar load operation and vector
splat operation. It first loads a scalar from the linear memory and then
broadcasts the value to all vector lanes. SableWasm uses a similar reduce rule
compared to the `zero-padding' load operation, except that instead of inserting
the scalar into a zero-initialized vector, we use \texttt{VectorSplat} to
broadcast it. The example above demonstrate this with
\texttt{v128.load32\_splat}. Similar to the `zero-padding' load operation,
`load-and-splat' is defined for all packed integers and packed float-point
numbers.

\paragraph{SIMD load lane} \quad
\begin{lstlisting}[
    basicstyle=\linespread{0.7}\small\ttfamily, 
    language=SableWasmMIR, 
    mathescape=true]
[..., %base i32, %vec v128]
v128.load32_lane offset=%offset align=%align lane=%lane $\Longrightarrow$
    %addr = int.add %base %offset
    memory.guard %mem %addr 4
    %t1 = load.32 i32 %mem %addr
    %t2 = v128.int.insert i32x4 %lane %base %t1
\end{lstlisting}
The next variation of the SIMD vector load operation is the `load-lane' load
operation. The example above demonstrates the procedure with a sample of
WebAssembly's \texttt{v128.load32\_lane} which reads a 32-bit integer from
linear memory and inserts it into a specific lane of a given vector. SableWasm
first lowers the load semantic using the same protocol as we discussed above and
then inserts to the given vector using the \texttt{VectorInsert} instruction.
Again, the WebAssembly SIMD extension proposal defines `load-lane' load
operation for all shapes of packed integers and floating-point numbers. In
WebAssembly SIMD load operation variations, one may already notice that we only
have a width associated with them instead of types. This is because WebAssembly
SIMD operations do not distinguish the shape of the vector. Hence, there is no
difference in loading a 32-bit integer and a single-precision floating number,
as they both consume 32-bit storage. But in SableWasm, we distinguish between
packed integers and packed floating-point numbers for the SIMD instruction shape
record. On the other hand, SableWasm also erases shape information from the
vector value, and it is the responsibility of the instruction to interpret the
value correctly. Thus, when we perform a load operation, we always assume that
we are loading packed integers. In the examples above, the 32-bit load with
translate to `load a 32-bit integer'.

\paragraph{SIMD load and extend} \quad
\begin{lstlisting}[
    basicstyle=\linespread{0.7}\small\ttfamily, 
    language=SableWasmMIR, 
    mathescape=true]
[..., %base i32] v128.load16x4_s offset=%offset align=%align $\Longrightarrow$
    %addr = int.add %base %offset
    memory.guard %mem %addr 8
    %t1 = load.64 v128 %addr 8
    %t2 = cast i32x4.extend.low.i16x8.s %t1
[..., %base i32] v128.load16x4_u offset=%offset align=%align $\Longrightarrow$
    %addr = int.add %base %offset
    memory.guard %mem %addr 8
    %t1 = load.64 v128 %addr 8
    %t2 = cast i32x4.extend.low.i16x8.u %t1
\end{lstlisting}
The last variation of a load operation is the `load-and-extend' load operation.
It is a combination of partial load and extension on the lower half of
128-bit vectors. In the example above we present examples for
\texttt{v128.load16x4\_s} and \texttt{v128.load16x4\_u}. The previous
instruction loads four 16-bit integers into the lower lanes of the vector and
performs sign-extension on the result to get a packed 32-bit integer vector.
\texttt{v128.load16x4\_u} performs a similar operation, except that it performs
zero-extension instead of sign-extension. A quick reminder, SableWasm MIR
\texttt{Load} instruction can apply to any primitive value type and supports
partial loading by annotating with a smaller load-width. In the case of the
partial load, SableWasm MIR \texttt{Load} always loads bytes starting from the
least significant bit and performs zero-extension on the result. SableWasm takes
advantage of the \texttt{Load} instruction's design when lowering the
`load-and-extend' load operation. In the example above, we partially load a
128-bit vector with a 64-bit value which corresponds to loading four
16-bit integers from the linear memory. Note that this \texttt{Load} instruction
yields a vector of 16-bit integers with four zero values in its higher lanes and
loaded values in its lower lanes. Thus, we only need to perform a \texttt{Cast}
operation with opcode \texttt{i32x4.extend.low.i16x8.s} to reach the desired
result. SableWasm treats \texttt{v128.load16x4\_u} using a similar procedure,
except that it uses zero-extension instead of sign-extension. Finally, like
other load operation variations discussed above, WebAssembly defines the
`load-and-extend' load operation for all packed integer and packed
floating-point numbers.

\paragraph{SIMD store lane} \quad
\begin{lstlisting}[
    basicstyle=\linespread{0.7}\small\ttfamily, 
    language=SableWasmMIR, 
    mathescape=true]
[..., %base i32, %val v128]
v128.store32_lane offset=%offset align=%align lane=%lane $\Longrightarrow$
    %addr = int.add %base %offset
    memory.guard %mem %addr 4
    %t1 = v128.int.extract i32x4 %val %lane
    store.32 %mem %addr %t1
\end{lstlisting}
Similar to the `load-lane' load operation variation, the WebAssembly SIMD
extension proposal also defines direct lane store instruction for 128-bit
vectors. The above example demonstrates the reduced rules for these
instructions. Let's take \texttt{v128.store32\_lane} as example. SableWasm MIR
first calculates the address and sets up a memory boundary check use a protocol
similar to what we have seen above. Then, it extracts the lane value by using
\texttt{VectorExtract} instruction and stores it into linear memory. Like
WebAssembly \texttt{load} instructions, the \texttt{store} instruction does
not distinguish between packed integers from packed floating-point numbers.
In SableWasm, we always assume the store vector is packed integers.


\section{Analysis Framework}
\label{section:mir-opt}

\begin{figure}
    \centering
    \includegraphics[width=\textwidth]{Images/4.MIR/analysis-framework.pdf}
    \caption{SableWasm MIR Analysis and Optimization Framework}
    \label{fig:sablewasm-mir-analysis-framework}
\end{figure}

SableWasm also implements an analysis and optimization framework over its
middle-level intermediate representation (MIR). The framework consists of two
parts, passes and drivers. The SableWasm analysis and transformation framework
only provides essential support for managing passes, compared to other more
advanced frameworks, such as McSAF\cite{mcsaf}, an optimization framework for
MATLAB language. Figure~\ref{fig:sablewasm-mir-analysis-framework} illustrates
the current state of the framework in SableWasm. Currently, we implement three
different drivers. \texttt{SimpleModulePassDriver} accepts module passes and
operates on the module level. At the time of thesis writing, we haven't explored
inter-procedural analysis for SableWasm MIR in detail, and the only module pass
implemented is the pretty-print pass. In the future, one can add additional
inter-procedural analyses to SableWasm, by implementing the \texttt{ModulePass}
interface. The second driver is the \texttt{SimpleFunctionPassDriver}. As its
name suggests, it manages \texttt{FunctionPass} instead. \texttt{FunctionPass}
implements intra-procedural analysis that operates over basic blocks. SableWasm
currently implements multiple intra-procedural analyses, such as dominator tree
construction. We will cover these passes in detail in this section. The last
driver in SableWasm is \texttt{SimpleForEachFunctionPassDriver} which is a
wrapper class for \texttt{SimpleFunctionPassDriver}. It works with
\texttt{FunctionPass} but takes a module as an argument.

\subsection{Dominators and Dependence}

Dominator tree and immediate dominance are close related to \emph{static single
    assignment} (SSA) form, and Ron Cytron's classic paper on converting
control flow graph (CFG) to SSA \cite{ibm-ssa} shows that SSA directly derives
from them. The dominator tree represents the dominance relationship between
basic blocks. A basic block is a \emph{dominator} of another if all control flow
reaching the later block must go through the first block. On the other hand,
\emph{immediate dominance} defines a stricter relationship between basic blocks.
A basic block is an immediate dominator of another if it satisfies two
conditions. First, the candidate block must be a dominator block of the second
one. Second, it does not dominate any other blocks that dominate the second
block. Although the SableWasm MIR is already in SSA form, the dominator tree is
still helpful in the later analysis and backend code generation. One may notice
that the dominator relationship in SSA is comparable to the same problem in
graph theory. Indeed, they are the same problem if we treat the basic blocks as
vertices and control flow paths as edges among them.  A direct solution to
compute the dominator set utilizes forward analysis within $O(n^2)$, respecting
to the number of basic blocks in the CFG. More efficient algorithms can yield
the dominator set within almost linear time, such as Tarjan's algorithm
\cite{tarjan-fast-dominator}, and its refined version
\cite{tarjan-fast-dominator-improved}. Currently, SableWasm compiles programs
usually with smaller functions that contain approximately 200 basic blocks at
most. Hence, an efficient complex algorithm does not have too much room for
improvement. In the future, if this becomes the bottleneck of the compilation
pipeline, one should replace the implementation with a better algorithm. This
section will present the forward analysis implementation briefly, and it is a
classic implementation for dominator tree construction.

\paragraph{Formalisms}
In the rest of the section, we will use $dom(\cdot)$ to represent the set of
\emph{strict dominators} for a given basic block. The set of strict dominators
for $A$ is the set of dominators for $A$ subtracting $A$ itself. Hence, `block
$A$ is a strict dominator for block $B$' implies that $A \in dom(B)$. Similarly,
$BB_{idom}$ is an immediate dominator for basic block $A$, if and only if,
$BB_{idom} \in dom(A)) \land (\forall B \in dom(A), BB_{idom} \notin dom(B)$.
Finally, the dominator tree represents all basic blocks with tree nodes and adds
directed edges according to the immediate dominator relationship.

\paragraph{Dataflow analysis}
The algorithm is a classic forward dataflow analysis. In this paragraph, we will
quickly cover the key points in the algorithm. For more detailed information,
one should consult Cytron's paper on SSA construction. During pass
initialization, we first set the following,
$\forall A \in BB \setminus \{ BB_{entry} \}, dom(A) = BB$,
where $BB$ denotes the set of all basic blocks that appeared in the control flow
graph, and $BB_{entry}$ denotes the entry basic block for CFG. For the entry
basic block, we set $dom(BB_{entry}) = \{ BB_{entry} \}$ instead. The
initialization value is a conservative guess of the result, and the next step is
to refine it. The iterative step rule is as follow,
$$
    \forall A \in BB, dom(A) =
    \left\{\{ A \} \cup \bigcap_{B \in pred(A)} dom(B)\right\}
$$
Here, $pred(\cdot)$ denotes the predecessor of the given basic block. The
general idea is that a basic block that dominates all its predecessors must also
dominate the given basic block for each of the basic blocks. The stop criteria
for the dominator analysis are also quite simple. If there are no more changes
in the result, the forward analysis will terminate.

\paragraph{Implementation}
SableWasm implements the forward dataflow analysis we discussed above with class
\texttt{DominatorPass}. In addition, the analysis pass object shares its result
with a helper class \texttt{DominatorPassResult} which provides helper
methods for accessing the result, such as calculating the immediate dominator
and constructing the dominator tree from the result sets. Finally, SableWasm
uses several techniques to improve the performance, such as modeling the set
with sorted arrays.

In this section, we presented the dominator analysis in SableWasm. The dominator
analysis is quite common among compiler implementations, and it will play a
critical role in the latter part of the project.
\subsection{Control-Flow Graph Simplification}

In section 4.2, we illustrated the translation rule from WebAssembly bytecode to
SableWasm MIR. Unfortunately, the translation rules yield suboptimal control
flow graphs. Hence, in this section, we will incrementally improve the control
flow graphs by fixing several obvious issues we found, such as trivial $\phi$
nodes and unnecessary branching. The control flow graph simplification also
performs \emph{dead code elimination} and \emph{unreachable basic block
    elimination}. This section presents the patterns, along with their
transforming strategies used in SableWasm. The general design of the
simplification pass is similar to what one would expect in a peephole optimizer
\cite{peephole-opt}. It iterates through the control flow graph, scans for
matched patterns, and if it finds any optimization opportunities it will apply
transformation strategies immediately. In the future, one may generalize this
simplification pass into a fully-featured peephole optimizer.Using a
domain-specific language for patterns similar to Alive
\cite{alive, alive-in-lean} for LLVM to ensure extensibility and correctness of
the patterns. The simplification pass will terminate once the execution reaches
a fixed point, where there are no more optimization opportunities.

\begin{figure}
    \begin{minipage}[t]{.5\textwidth}
        \lstinputlisting[
            basicstyle=\linespread{0.7}\footnotesize\ttfamily,
            language=SableWasmMIR,numbers=left
        ]{Code/4.MIR/simplify-cfg.mir}
    \end{minipage}\hfill
    \begin{minipage}[t]{.5\textwidth}
        \lstinputlisting[
            basicstyle=\linespread{0.7}\footnotesize\ttfamily,
            numbers=left
        ]{Code/4.MIR/simplify-cfg.wat}
    \end{minipage}
    \caption{Control-flow graph simplification example}
    \label{fig:simplify-example}
\end{figure}

\paragraph{Trivial $\phi$ nodes}
The first pattern we found in generated SableWasm MIR is the trivial $\phi$
nodes. Trivial $\phi$ nodes refer to the $\phi$ nodes with only one candidate
value. In section 4.2.1, we present the translation patterns for \texttt{loop}
instructions in WebAssembly and mentioned that the pattern is suboptimal and
will result in trivial $\phi$ nodes. A quick reminder, the \texttt{loop}
instruction needs to insert $\phi$ nodes to the landing BB, which necessarily
has non-merging control flow as an effect of a limitation in our translation
framework. To address this, we search for \texttt{\%t0 = phi t [\%t1, \%path]}
for all possible type $t$. The transformation strategy is to replace all
appearances of value \texttt{\%t0} with value \texttt{\%t1}. As the $\phi$ nodes
do not map to any operations and are only introduced by SSA to explicitly mark
value merging, removing them from the control flow graph does not change the
semantics of the program. When replacing the values, SableWasm uses the use-site
lists managed by the \texttt{ASTNode} to boost the performance.

\paragraph{Redundant branching}
The second pattern focus on redundant branching. Redundant branching can also
come from the translation patterns for structured control flow. One may already
notice that we will always generate a landing basic block for the instruction
for every structured control flow construct. However, when the control flow
constructs are the last instructions in their enclosing expression, the landing
basic blocks will only contain a single branching instruction.
Figure~\ref{fig:simplify-example} demonstrates an unoptimized example. On the
right-hand side, the WebAssembly function is a simple function that returns one
when the operand is an even number and zero otherwise. On the left-hand side is
its corresponding SableWasm MIR before simplification. Clearly, \texttt{\%BB:0}
and \texttt{\%BB:1} are redundant. The redundant branch elimination pattern
looks for basic blocks with a single inward flow and attempts to merge them
with their predecessors. In the example, the optimizer will try to merge
\texttt{\%BB:1} and \texttt{\%BB:4} by moving the \texttt{Constant} instruction
into \texttt{\%BB:1}, and redirecting the branching in \texttt{\%BB:1} from
\texttt{\%BB:4} to \texttt{\%exit}.

\paragraph{Dead basic block}
The third pattern we have in SableWasm to simplify control flow graph is dead
basic block elimination. In figure~\ref{fig:simplify-example}, we have a dead
basic block, namely \texttt{\%BB:2}. These dead basic blocks again come from
SableWasm's translation patterns. When we are translating the control flow
constructs, we always prepare the landing basic block. However, in many cases,
the control flow may not reach the landing basic block. In the example above,
we have a WebAssembly \texttt{return} instruction appear in the \texttt{block}'s
nested expression. The translation patterns for \texttt{return} instruction is
naive, which creates a branch to the exiting block and configures the $\phi$
nodes accordingly. Hence, in this case, the landing basic block will never have
an inward flow. In SableWasm MIR, we do not consider these unreachable basic
blocks malformed. However, in many backends, these are considered bad behaviour.
In addition, these basic blocks also interfere with other optimizations. In the
example in figure~\ref{fig:simplify-example}, \texttt{\%BB:3} does not satisfy
the redundant branching elimination pattern because it does not have a unique
inward flow. However, one of them, \texttt{\%BB:2}, is a dead block. Thus, by
removing dead basic blocks from the control flow graph, we may find more
optimization opportunities. In SableWasm, we identify the dead basic block via
a mark-and-sweep algorithm. Starting from the entry block, we mark all the basic
blocks that are reachable. Then we iterate overall basic blocks, and if the
basic block does not have the flag, we add them to the delete list. Finally, we
remove all the basic blocks within the delete list from the control flow graph.

\begin{figure}
    \lstinputlisting[
        basicstyle=\linespread{0.7}\small\ttfamily,
        language=SableWasmMIR,numbers=left
    ]{Code/4.MIR/simplify-cfg-result.mir}
    \caption{Control-flow graph simplification result}
    \label{fig:simplify-result}
\end{figure}

\paragraph{Dead value}
The last pattern we have in the control flow graph simplification pass is dead
value elimination. Dead value elimination is similar to the dead basic block
elimination, except that it works with values instead of basic blocks.
Unfortunately, the example in figure~\ref{fig:simplify-example} does not contain
any dead values. However, the idea is quite simple to understand. Most of the
dead values come from WebAssembly's \texttt{drop} instruction which discards
values from the implicit operand stack. In a non-SSA control flow graph, one
usually needs first to perform \emph{liveness analysis} and \emph{reaching
    definition analysis} to determine if the value is dead. But in SSA, one can
quickly recover this information from use-definition chain, and in SableWasm,
the base class \texttt{ASTNode} automatically manages it. Thus, the optimizer
will iterate over all values within the control flow graph and check if others
refer to it. If not, it then verifies if the instruction is \emph{droppable}.
A droppable instruction is an instruction such that if we remove it from the
control flow graphs, no observable effects should happen, similar to the
concept of `pure' for functions. Finally, if instructions are both dead and
droppable, the optimizer will remove them from the control flow graph.

In this section, we covered the flow graph simplification pass in SableWasm. The
optimizer will iteratively run four patterns that we have discussed above until
it reaches a fixed point. Figure~\ref{fig:simplify-result} shows the result of
running these optimizations on the input shown in
figure~\ref{fig:simplify-example}. Compared to the original, the result is more
readable. Moreover, by reducing the number of basic blocks, we can improve other
analyses in SableWasm.

\subsection{Type Inference}
\label{section:mir-opt-type-inference}

This section presents the type system for SableWasm MIR. SableWasm MIR is a
statically typed language with a pretty straightforward type system. However,
one may already notice that SableWasm MIR does not annotate every instruction
with a type, unlike many other compiler intermediate representations. Instead,
SableWasm computes the types for values on-demand via a set of type inference
rules. The type system for SableWasm MIR generalizes from the MVP WebAssembly
type system and its extension proposals with a few modifications. The
formal definition for SableWasm MIR types are as follow,

\begin{lstlisting}[basicstyle=\linespread{1}\ttfamily, mathescape=true]

$\langle$primitive_type$\rangle$ ::= i32 | i64 | f32 | f64 | v128
$\langle$tuple_type$\rangle$     ::= (N, $\langle$primitive_type$\rangle$$\dots$)
$\langle$type$\rangle$           ::= $\langle$primitive_type$\rangle$ | $\langle$tuple_type$\rangle$ | () | $\bot$

\end{lstlisting}

Here we will skip the discussion for \emph{primitive type} and the type checking
rules for its corresponding instructions as they are equivalent to the MVP
WebAssembly type system. The \emph{tuple type} consists of an unsigned integer
and a list of primitive types. They model the return types of multi-value return
functions or \texttt{Pack} instructions. Finally, we introduce the unit type,
$()$, and the bottom type, $\bot$. One can consider the unit type as
\texttt{void} in the C programming language. It represents no value present,
but the type is valid. On the other hand, the bottom type, $\bot$, signals that
the pass cannot assign any valid type to the term. In the rest of this section,
we will focus our discussion on extensions made due to two major WebAssembly
extension proposals, multi-value and SIMD operation.

\paragraph{Multi-value}
WebAssembly multi-value extensions allow functions to have more than one return
values, which is quite interesting. Usually, low-level bytecode representation
does not directly support this feature, and it usually only appears in
higher-level language designs, such as Python. In
section~\ref{section:mir-design-insts}, we introduced
two instructions \texttt{Pack} and \texttt{Unpack}, along with how we represent
multi-value for functions. As a quick recap, SableWasm uses tuples to denote the
multi-value return for functions. The \texttt{Pack} instruction collects values
and constructs a tuple containing them, while on the other hand, the
\texttt{Unpack} extracts primitive values from tuples. Let's focus on the
\texttt{Pack} instruction first. The typing rule for \texttt{Pack} is
straightforward. If we can infer types for all candidate values, we say that the
\texttt{Unpack} instruction has a tuple type consisting of the number of
candidate values and a list of element types. On the other hand, if any of the
candidate values result in a non-primitive type, the \texttt{Pack} instruction
is the $\bot$ type. More formally,
$$
    \frac
    {\Gamma \vdash v_0 \Rightarrow t_0, \dots, v_n \Rightarrow t_n \qquad \forall i, t_i \in primitives}
    {\Gamma \vdash \text{\textbf{pack} } v_0, \dots, v_n \Rightarrow \langle n, t_0 \dots t_n \rangle}
    \qquad
    \frac
    {\Gamma \vdash \exists i, v_t \notin primitives}
    {\Gamma \vdash \text{\textbf{pack} } v_0, \dots, v_n \Rightarrow \bot}
$$
Here the set $primitives$ is the set of all possible primitive types in the
SableWasm MIR type system. For \texttt{Unpack} instructions, the type checker
will first check if the immediate index is within the tuple size. If the index
is out of bounds, the type checker will assign the instruction with bottom type
$\bot$. Otherwise, it will take the type from the tuple specified by the index.
Formally,
$$
    \frac
    {\Gamma \vdash v \Rightarrow \langle n, t_0 \dots t_n \rangle \qquad 0 \leq k \leq n}
    {\Gamma \vdash \text{\textbf{unpack } k v} \Rightarrow t_k}
    \qquad
    \frac
    {\Gamma \vdash v \Rightarrow \langle n, t_0 \dots t_n \rangle \qquad \text{otherwise}}
    {\Gamma \vdash \text{\textbf{unpack } k v} \Rightarrow \bot}
$$
We also generalize the function type in WebAssembly so that SableWasm MIR's
function type will always have a single return value. We use the following
strategy to map WebAssembly's function type into SableWasm MIR function type.
In the case where there are no return values, we translate the return type into
unit type. For example, SableWasm translate \texttt{[i32] -> []} into
\texttt{[i32] -> ()}. On the other hand, if the function type has exactly one
return value, the translation rule is trivial. Finally, when there are multiple
return values, we pack them into a single tuple. For example, SableWasm use
\texttt{[i32] -> (2, i32, f32)} to represent \texttt{[i32] -> [i32, f32]} in
WebAssembly.

\paragraph{SIMD operations}
Section~\ref{section:mir-design-insts} presented the instruction design in
SableWasm MIR. We mentioned
that WebAssembly's 128-bit vector value, added by the SIMD operation extension
proposal, does not store their shape information in the type. WebAssembly's
design gives us two choices in SableWasm when designing a type system for vector
operations. First, we can erase all the shape information for values and
carefully plan the instruction semantics to ensure that all the operations
have defined behaviour at runtime. Second, another approach is to add shape
information back to the values' types. If there is a mismatch in shape
information, either the translation visitor can insert a bit cast, or the type
checker can reject the program. In SableWasm MIR, we take the first approach by
erasing all the shape information from the vector values.
Chapter~\ref{chapter:backend-and-runtime} will introduce the second approach in
detail. The semantics for SIMD instructions in SableWasm MIR follows the
WebAssembly's specification. We always store the value using the little-endian
method and the vectors start their first lane from the least significant bit.

In this section, we talked about the type inference pass in SableWasm MIR.
Similar to the dominator analysis we seen in
section~\ref{section:mir-opt-dominator}, the type infer pass
does not optimize the control flow graph. But they are critical in the backend
when we lower the SableWasm MIR into LLVM. We will come back to this in detail
in chapter~\ref{chapter:backend-and-runtime}.