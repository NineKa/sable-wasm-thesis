\chapter{Related Work}

In previous chapters, we present the SableWasm and evaluate its performance against several well-known benchmark suites. SableWasm is not the only WebAssembly runtime environment system that allows sandboxed WebAssembly modules to run outside the browser. This chapter will provide a quick overview of several existing WebAssembly host environment implementations. Also, SableWasm does not implement any auto-vectorization algorithms and heavily depend on LLVM, both in the frontend WASI-enable Clang compiler and the SableWasm backend, to generate parallel code. Auto-vectorization is one of the key research fields in compiler optimization, and much research has been devoted to the field. Therefore, we will briefly cover auto-vectorization in LLVM in this chapter.

\section*{WebAssembly runtime environments}

In chapter 6, we mentioned two WebAssembly runtime environments developed by the community, Wasmtime and Wasmer. Wasmtime perhaps is the earliest non-browser WebAssembly runtime environment. It started as a side project during WebAssembly standardization and is maintained by Bytecode Alliance\footnote{Bytecode Alliance: \url{https://bytecodealliance.org/}}. This cross-industry nonprofit organization focuses on extending WebAssembly and WASI beyond the browser and IoT devices. Wasmtime is built on the Cranelift compiler framework\footnote{Cranelift:\url{https://github.com/bytecodealliance/cranelift}}. Cranelift is similar to LLVM, providing a target-independent intermediate representation that eventually translates to native executable machine code. Currently, at the time of thesis writing, Cranelift is still at very early stages and only supports the x86-64 target. Although the Cranelift started as the backend for Wasmtime, it is not limited to the Wasmtime project. In the future, Cranelift may replace the fast debug backend in the Rust compiler toolchain and the Javascript/WebAssembly engine backend in SpiderMonkey.

Wasmer \footnote{Wasmer:\url{https://wasmer.io/}} is another WebAssembly runtime environment and maintained by a startup company. Wasmer shares many similarities comparing to Wasmtime. However, it is more flexible in design. Currently, Wasmer has three different backends, LLVM, Cranelift and a single-pass compiler for fast code generation. Additionally, comparing to Wasmtime, Wasmer is more aggressive in adding features to WebAssembly. For example, Wasmtime only supports WASI as the system interface API, while Wasmer supports both WASI and Emscripten specification. Wasmer also comes with a package manager, called WebAssembly Package Manager (WAPM) \footnote{WebAssembly Package Manager (WAPM): \url{https://wapm.io/}} which distributed pre-compiled sandboxed WebAssembly binary modules for various applications.

Wasmtime and Wasmer are both just-in-time (JIT) WebAssembly runtime environments. There are also ahead-of-time (AOT) compilers for WebAssembly modules. The most notable one is perhaps the Lucet compiler. Lucet\footnote{Lucet: \url{https://www.fastlylabs.com/}} is developed by Fastly and shares a similar design as SableWasm. The initial motivation for Lucet is to create a cloud application system that hosts user-uploaded WebAssembly modules. Currently, Lucet powers Fastly's Terrarium platform, an in-browser multi-language IDE. The Lucet compiler system has two parts, the Lucet shared library compiler, and the Lucet shared library loader. The Lucet shared library compiler compiles WebAssembly modules into shared libraries, while Lucet shared library load dynamically loads the shared library and executes the entry function, \texttt{\_start}. Unlike SableWasm, Lucet is also built on the Cranelift compile framework.

All the WebAssembly environments we have discussed in the section are built on complex compiler frameworks such as Cranelift or LLVM. Therefore, one question that arises naturally is whether WebAssembly is suitable in a resource-constrained environment such as an embedded system. It turns out that it is possible to translate WebAssembly bytecode, under these conditions, into native executable code while maintaining decent performance, as shown in the paper \cite{webassembly-embedded}. One interesting application for WebAssembly is to use it as a form of distributing programs on IoT devices. For example, in this paper \cite{webassembly-wearables}, researchers implement a WebAssembly interpreter on an SoC that communicates and receives modules from a host device. The system runs on low-power Bluetooth, and in theory, can be used on a wearable device. Another WebAssembly runtime system, Twine, presented in paper \cite{webassembly-sgx}, focuses on taking advantage of hardware features to further improve the performance of WebAssembly. For example, Twine takes advantage of the Intel SGX instruction set to ensure the module's security and achieve up to 4.1x speedup in performance.

\section*{Auto-vectorization}

This section will briefly discuss auto-vectorization in compilers, more specifically, the LLVM compiler framework. Modern CPU architectures support vector operations to some degree, such as SSE \cite{sse-intel}, AVX \cite{avx-intel} on x86, and Neon \cite{arm-neon} on Arm. In recent years, scalable vector extension, such as Arm's SVE \cite{arm-sve}, offers even more flexibility on vector size. Although these SIMD instruction set extensions speed up the resulting program, programmers need to have an in-depth understanding of the hardware system to handle them correctly through inlined assembly or intrinsic functions. Additionally, these methods are highly hardware-specific and cause trouble when porting programs to another platform. Another approach to the problem is to ask the compiler to generate vectorized code from traditional scalar codes, hence the name auto-vectorization, which is implemented in many modern compiler systems, such as GCC \cite{auto-vec-gcc} and LLVM \footnote{Auto-vectorization in LLVM: \url{https://llvm.org/docs/Vectorizers.html}}. Here we will take the LLVM auto-vectorizer as an example.

The first attempt for auto-vectorization in LLVM is the basic block vectorizer. It works with a single basic block at a time and searches for common patterns. If it finds any optimization opportunity, it will rewrite the basic block into parallel form. One might notice that the basic block vectorizer has no understanding of a loop structure and only perform auto-vectorization if and only if the operations are already unrolled. To address this problem, the second generation of auto-vectorizer is a single block loop vectorizer. The single block loop vectorizer can recognize simple loop structures and consists of two parts the loop legalizer and the loop transformer. The loop legalizer determines if a loop structure can undergo auto-vectorization, and if so, the loop transformer will perform the rewrite. The single block loop auto-vectorizer can also perform loop unrolling if the induction variables are detected. However, this sometimes leads to very aggressive optimization, which slows down the generated code. Hence, in late 2012, the LLVM developers extended the auto-vectorizer with a cost model \cite{llvm-vec-cost, llvm-vec-cost-avx}. The cost model will determine whether a potential optimization worth it based on the instruction set available on the target hardware and data dependency between operands.  

LLVM also performs another type of auto-vectorization called superword-level parallelism (SLP) auto-vectorization. SLP auto-vectorization combines similar operations into vector operations, such as memory access and numerical comparison. SLP auto-vectorization is similar to the basic block auto-vectorizer discussed earlier in this section, except that it searches patterns in a bottom-up fashion. 

Although auto-vectorization brings a silver lining to systemically transforming scalar code into parallel form, it still suffers several drawbacks. The most notable problem is that the dependency between instructions is usually not apparent to the compiler, especially in a nested loop structure. Hence, the compiler can only take a conservative approach when scheduling the program. One possible solution is to employ a polyhedral model \cite{polyhedral} to analyze the data dependency variables. The polyhedral analysis creates polyhedra based on the program and applies affine transformations to improve instruction scheduling incrementally. LLVM implements the polyhedral analysis in the project Polly \cite{polly} \footnote{LLVM Polly: \url{https://polly.llvm.org/}}, which can be used as a compiler plugin and generates scheduling and scope information for instructions. Later, the LLVM loop auto-vectorizer can take advantage of this information to provide better cost estimation. 