\chapter{Future Work and Conclusion}

WebAssembly has been growing in popularity in recent years as a new format
for distributing sandboxed applications over the internet. In this project, we
presented SableWasm as a standalone ahead-of-time (AOT) compiler for
translating WebAssembly modules into shared libraries. We also implement a
runtime environment that enables other programming languages such as C/C++ to
interact with generated shared libraries.

We first started the project with a custom parser for WebAssembly binary format.
The parser focuses on extensibility and performance, as currently, WebAssembly
is still under the standardization process and several syntax extensions might
be merged to the specification soon. We then evaluate the performance of the
parser by benchmarking against \texttt{wabt}, the reference implementation
provided by the WebAssembly community, and observe a 1.6x speedup in execution
speed and a 4.6x reduction in memory footprint.

We then define the middle-level representation (MIR) for SableWasm. SableWasm
MIR is a register-based control flow graph representation of the WebAssembly
program. When translating WebAssembly bytecode to SableWasm MIR, we focus on
two significant problems. First, WebAssembly is a stack-based bytecode with
structured control flow structures. Thus, to faithfully translate the bytecode,
we define translation patterns that mimic the semantics of these constructs and
reduce them into basic blocks and branching instructions. The other problem we
encountered is regarding the size of the instruction set. As WebAssembly encodes
both type and shape information in the instruction opcode, the WebAssembly
instruction set is quite large. Additionally, to reduce the size of the module,
WebAssembly fuses several typical instruction sequences into one single
instruction, such as load-and-extend. To maintain a small instruction set, we
define several reduction rules for WebAssembly instructions. Unfortunately,
these translation patterns lead to awkward and inefficient code. To address
this problem, we implement an analysis and transformation framework over
SableWasm MIR. We then design simplifying control flow graph pass that
incrementally improves the MIR, similar to a `peephole optimization' by
locating and replacing several common redundant patterns.

The last component of SableWasm is the SableWasm runtime library, which provides
implementations for builtin functions used in the generated shared libraries.
It also implements several WebAssembly entities, such as the linear memory and
the indirect table. Currently, the SableWasm runtime library defines an
easy-to-use C/C++ interface to the user and handles errors and exceptions
using the exception mechanism in C++.

Finally, we evaluate SableWasm's performance by benchmarking with three
well-known benchmark suites, Polybench, Ostrich, and NPB. The first question
we focus on in this thesis is how SableWasm performs compared to other existing
WebAssembly runtime environments. We conclude that SableWasm performs on par
with Wasmtime and approximately 1.5x to 2x faster than Wasmer. The second
research question is whether optimization over input WebAssembly modules
affects the overall performance in SableWasm. By comparing the execution time
of SableWasm under optimized translated input modules against that of naive
translated input modules, we conclude that, currently, optimization over the
WebAssembly modules has a significant impact on the performance. Hence, when
designing frontend compilers that target WebAssembly, one should be careful of
translation patterns and perform optimizations as early as possible. The last
question we investigated in this thesis is whether the WebAssembly SIMD
extension brings performance improvement to SableWasm. Experimentally, we can
see significant benefit. However, using Polybench, we locate many common
patterns that cannot be identified and rewritten by LLVM's auto-vectorizer
in SableWasm.

\section*{Future Work}

WebAssembly is a relatively new language, and many of its features are still
under the standardization phase. SableWasm only covers several WebAssembly
extensions such as the multi-value extension and the SIMD operation extension.
One excellent opportunity is to implement more WebAssembly extensions in
SableWasm, such as the garbage collection (GC) extension and the exception
handling extension. Currently, many high-level languages, such as
AssemblyScript, require static linking with a non-trivial runtime library when
cross-compiling into WebAssembly. Simulating these features results in notable
increases in code size and a slow down in performance.

Another interesting direction is to add more analysis and transformation in
SableWasm under the optimization framework. More specifically, one can implement
an auto-vectorizer in SableWasm at the MIR level. The evaluation chapter shows
that the LLVM's auto-vectorizer cannot recognize many apparent patterns and
yields inefficient code. We suspect that the boilerplate code introduced by
the translation patterns confuses the auto-vectorizer. Hence, an auto-vectorizer
at the SableWasm MIR level can better understand the program and, in theory,
recover more opportunities within the WebAssembly modules.

Finally, one can also add more backend support for SableWasm. Currently,
SableWasm is an ahead-of-time (AOT) compiler built on the LLVM compiler
infrastructure. One natural extension of this project to implement a
just-in-time (JIT) system that uses LLVM's Orc JIT framework. Additionally,
one can also explore many profile-guided optimizations (PGO) techniques used
in many other VM languages, such as Java bytecode \cite{java-pgo}.