\section{Interface with C/C++}
The last section of the chapter will cover the interface between the generated shared library and the host languages. Currently, SableWasm only has a binder library for C/C++. However, the principle is relatively straightforward, and one can add implements binder function for any other languages. In the rest of the section, we will focus our discussion on the callee wrapper, WASI function implementations and error handling strategies.

\paragraph{Callee wrapper}
Section 5.2 mentioned that SableWasm stores function instance as a pair of context pointer and function address pointer. Additionally, SableWasm also encodes the function types as null-terminated strings. However, all this information is only available to the host program at runtime. C/C++ is a statically typed language; hence, we can only specify type contracts on the exported functions at compile-time and verify the contracts at runtime. Traditionally, one can use a type erased pointer, a \texttt{void} pointer, to store the function address and reinterpret it to the actual concrete type.  SableWasm presents a helper class that provides type-safe access to the exported functions, \texttt{WebAssemblyCallee}. \texttt{WebAsssemblyCallee} takes advantage of the template metaprogramming system in C++ and generates null-terminated encoding of expected type at compile-time. At runtime, the wrapper class will check the type signature string against the actual type string before forwarding the function call. If the type signature string mismatch, the system will signal an exception.

\paragraph{WASI interface implementation}
WebAssembly System Interface (WASI) extends the WebAssembly by providing syscalls that interact with the host environment. This extension is non-invasive, and all the syscalls are in the form of imported functions, mainly host functions. Hence, SableWasm implements the WASI extension using host library functions only. At the shared library initialization phase, the loader will set up WASI host functions based on the import descriptor. Currently, SableWasm only implements minimal WASI interface functions to run benchmarks, such as standard I/O and timing. However, the framework is easy to extend, and all the WASI function implementations are under the namespace \texttt{runtime::wasi}. Therefore, we will skip them in detail in the thesis; one can consult the source code for implementation detail of WASI interface functions. One of the project's future work is to continuously work on the WASI system interface and add more features to SableWasm, such as capability-based file system and networking.

\paragraph{Error handling strategies}

The last topic we will in the section is error handling. SableWasm builds its error handling strategy based on the C++ exception mechanism. Comparing to other exception handling strategies, this brings us two significant benefits. First, when generating LLVM intermediate representation for shared libraries, we can avoid boilerplate code that propagates exceptions. Additionally, on most modern system ABI that supports zero-cost exception handling, this gives SableWasm performance advantages.  On the other hand, this leaves us room for further improvement for pending WebAssembly extensions, such as WebAssembly exception handling extension \footnote{WebAssembly exception handling: \url{https://github.com/WebAssembly/exception-handling}}. WebAssembly exception handling extension generalizes WebAssembly specification by adding \texttt{try catch} construct to the syntax, which directly corresponds to the C++ exception handling mechanism.

\begin{figure}
    \centering
    \lstinputlisting[language=C++, basicstyle=\linespread{0.9}\small\ttfamily, numbers=left]
    {Code/Tester.cc}
    \caption{Simple C++ SableWasm loader function}
    \label{fig:sablewasm-loader}
\end{figure}

In this section, we discuss the interaction between C/C++ and SableWasm system. Finally, we will conclude the chapter with a concrete loader function example. Figure~\ref{fig:sablewasm-loader} demonstrates a simple loader function for generated SableWasm shared libraries. In the example above, we assume the WebAssembly module is a WASI compatible module, and hence, exports a function named \texttt{\_start} as the entry function with type \texttt{[] -> []}.
