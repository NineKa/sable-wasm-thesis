\section{Code Generation}

This section describes the code generation strategy used in the SableWasm LLVM backend. For most of the instructions, especially for SableWasm MIR numeric operations, the translation rules are simple mapping between SableWasm MIR instructions to their LLVM counterparts. In this section, we will skip the discussion over these trivial mapping. Instead, one can consult the SableWasm source code for more details. The rest of the section will focus on several key aspects: local variable implementation, linear memory manipulation, indirect function call, and SIMD instruction operations. One problem that arises when lowering SableWasm MIR into LLVM intermediate representation is how to pick the instruction translation order. Any instruction in SableWasm MIR can refer to values either generated by a previous instruction in the same basic block or instruction within a dominating block, implying that when lower SableWasm MIR, we need to perform a pre-order tree traversal over the dominator tree. However, Phi nodes only require the candidate value from a single inward flow instead of dominating the enclosing basic block. Hence, the translation visitor may not have translated the candidate value before Phi nodes. SableWasm backend takes a two-phase translation to address this problem. In the first pass, the backend will translate all the instructions and collect the resulting values into a map, and in the second pass, the backend will come back to the Phi Nodes and setup up the candidate values accordingly.

\paragraph{Function declaration and local variables} \quad
\begin{lstlisting}[basicstyle=\linespread{0.9}\small\ttfamily, language=LLVM, mathescape=true]
function %foo: [i32] -> [f32] {
  {(arg) %local0: i32, %local1: f64} 
  ......
}
$\Longrightarrow$
define private float @foo(%__sable_instance_t* %0, i32 %1) {
entry:
  %2 = alloca i32, align 4
  store i32 %1, i32* %2, align 4
  %3 = alloca double, align 8
  store double 0.000000e+00, double* %3, align 8
  ......
}

{%local: i32} 
%t0 = local.get %local $\Longrightarrow$ %t0 = load i32, i32* %local, align 4
local.set %local %t0 $\Longrightarrow$ store i32 %t0, i32* %local, align 4
\end{lstlisting}

We will first start by examining the translation pattern for lowering SableWasm MIR functions into LLVM functions and their local variables. The example above presents a simple function named \texttt{foo}, which takes a single 32-bit integer as the argument and returns a single-precision floating-pointer number. \texttt{foo} has two local variables. The parameter implicitly introduces the first one, \texttt{local0}, and the function explicitly defines the second one, \texttt{local1}. At runtime, \texttt{local0} will hold the value of the parameter upon entry, and \texttt{local1} will initialize to zero. Compare to the SableWasm MIR function definition, the one in LLVM intermediate representation (IR) has two major differences. First, the LLVM function definition has the extra instance object pointer in the arguments, in the example above, \texttt{\%0}. We covered this briefly in the instance layout section. In short, for all the functions, the SableWasm backend code generator will implicitly add the instance object pointer as the first argument. The other difference is in the entry block. SableWasm MIR, similar to WebAssembly, views the local variables as opaque memory slots. However, LLVM IR requires users to manually allocate them in stack memory space via \texttt{alloca} instruction. The \texttt{alloca} instruction reserves enough memory on the stack based on the given type and returns a pointer. In example above, \texttt{\%2} and \texttt{\%3} are two reserved local variable memory region that correspond to \texttt{local0} and \texttt{local1} accordingly. Another difference is that SableWasm IR defines implicit initialization for all local variables; on the other hand, LLVM \texttt{alloca} instruction leaves the reserved memory with uninitialized values. Hence, to faithfully implement WebAssembly and SableWasm MIR specification, we generate \texttt{store} instructions to set the initial values for each local variables. As for \texttt{LocalGet} and \texttt{LocalSet} instructions, the translation patterns are quite straightforward. SableWasm backend code generator maps \texttt{LocalGet} instructions to \texttt{load} instructions and \texttt{LocalSet} instructions to \texttt{store} instructions as demonstrated in the example above.

\paragraph{Linear memory operation} \quad
\begin{lstlisting}[basicstyle=\linespread{0.9}\small\ttfamily, language=LLVM, mathescape=true]
$\text{\textbf{Fetching linear memory:}}$
%t0     = getelementptr 
            inbounds %__sable_instance_t, %__sable_instance_t* %0, 
            i32 0, i32 4
%memory = load %__sable_memory_t*, %__sable_memory_t** %t0, align 8

%t0 = memory.size %mem $\Longrightarrow$
%t0 = call i32 @__sable_memory_size(%__sable_memory_t* %mem)
%t0 = memory.grow %mem %delta $\Longrightarrow$
%t0 = call i32 @__sable_memory_grow(%__sable_memory_t* %mem, i32 %delta)
memory.guard %mem %offset $\Longrightarrow$
call void @__sable_memory_guard(%__sable_memory_t* %mem, i32 %offset)
\end{lstlisting}
In section 5.1 and 5.2, we presented the instance object manages linear memory instance and several runtime functions that implement additional functionalities. The SableWasm backend code generator takes advantage of the design by mapping SableWasm linear memory manipulation instructions into built-in function invocations. The example above demonstrates the mapping for \texttt{MemorySize}, \texttt{MemoryGrow} and \texttt{MemoryGuard} instructions. All these instructions map to \texttt{call} instructions to their corresponding built-in functions with appropriate arguments. Note that all built-in functions require to pass the pointer the linear memory pointer as an argument. Currently, the WebAssembly module can have at most one linear memory. Due to the validation rules, such linear memory must present within the module if linear memory manipulation instructions appear in the program. Further, as we store linear memory instance pointers before any other entities, one can show that the linear pointer must be the 5th pointers in the instance object. Hence, SableWasm backend code fetch the linear memory instance pointer using a pair of a \texttt{getelementptr} instruction and a \texttt{load} instruction. The \texttt{getelementptr} instruction LLVM calculate the address for entries in a aggregation. The above example calculates the address base on the type \texttt{\_\_sable\_instance\_t} which is generated based on declared entities at compile time.

\paragraph{Linear memory load and store} \quad
\begin{lstlisting}[basicstyle=\linespread{0.9}\small\ttfamily, language=LLVM, mathescape=true]
$\text{\textbf{Load a 32-bit integer:}}$
%result = load.32 i32 %mem %addr $\Longrightarrow$
  %t0     = ptrtoint %__sable_memory_t* %memory to i64
  %t1     = zext i32 %offset to i64
  %t2     = add nuw i64 %t0, %t1
  %addr   = inttoptr i64 %t2 to i32*
  %result = load i32, i32* %addr, align 1
$\text{\textbf{Partial load a 32-bit integer:}}$
%result = load.16 i32 %mem %addr $\Longrightarrow$
  ......
  %t0     = load i16, i16* %addr, align 1
  %result = zext i16 %t0 to i32
$\text{\textbf{Store a 32-bit integer:}}$
store.32 %mem %addr %val $\Longrightarrow$
  ......
  store i32 %val, i32* %addr, align 1
$\text{\textbf{Partial store a 32-bit integer:}}$
store.16 %mem %addr %val $\Longrightarrow$
  ...... 
  %t0    = trunc i32 %val to i16
  store i16 %t0, i16* %addr, align 1
\end{lstlisting}
SableWasm MIR classify load and store instructions into two groups, partial and complete. A quick reminder, WebAssembly associates load and store operations with sign extension mode, while in SableWasm, we define load instruction perform zero extension, and store instruction always apply bit truncation. The first example above presents a complete load operation for a 32-bit integer. The translation pattern is relatively straightforward. Note that the linear memory instance pointer pointers to the first byte within the linear memory. Hence, the SableWasm backend code generator will first calculate the native write address by summing up offset and base pointer and map the \texttt{Load} instruction to \texttt{load} in LLVM. LLVM memory operation, such as \texttt{load} and \texttt{store} has a complementary attribute, \texttt{align}. In the background section, we introduced the attributes in LLVM. In short, \texttt{align} attribute marks an alignment requirement for memory access operations. As WebAssembly linear memory is comparable to a byte array, which read-write can occur at any point, we can only conservatively set the alignment to one that limits the LLVM backend instruction selector from generating instructions with alignment assumption. This, in theory, leads to less efficient code. However, later in the evaluation section, we determine this is not a bottleneck of the entire implementation. In the future, one can further improve the performance of SableWasm by designing analyses that infer lower bounds for alignment. The second example above demonstrates the translation pattern for partial load operation. Compare to the complete load instruction, the translation pattern for partial load instruction has an additional zero-extending operation, \texttt{zext} at the bottom, to implement the SableWasm MIR partial load semantics. On the other hand, the translation pattern for both complete and partial \texttt{store} instructions are very similar to \texttt{load} instructions. The most notable difference is the \texttt{trunc} instruction in partial \texttt{store}'s translation pattern which performs bit truncation on the operand.

\paragraph{Indirect function call} \quad
\begin{lstlisting}[basicstyle=\linespread{0.9}\small\ttfamily, language=LLVM, mathescape=true]
call.indirect %table %index %expect_ty $\Longrightarrow$ 
  call void @__sable_table_guard(%__sable_table_t* %table, i32 %index)
  call void @__sable_table_check(
    %__sable_table_t* %table, i32 %index, i8* %expect_ty)
  %t0 = call %__sable_instance_t* @__sable_table_context(
    %__sable_table_t* %table, i32 %index)
  %t1 = call %__sable_function_t* @__sable_table_function(
    %__sable_table_t* %table, i32 %index)
  %t2 = icmp eq %__sable_instance_t* %t0, null
  %t3 = select i1 %t2, %__sable_instance_t* %0, %__sable_instance_t* %t0
  %t4 = bitcast %__sable_function_t* %276 to ......
  %t5 = call ...... %t4(%__sable_instance_t* %t3, ......)
\end{lstlisting}
The SableWasm backend code generator implements indirect function call via a series of built-in function invocations. We have already presented the built-in function in section 5.2; hence, we will not show them in detail in this paragraph. The first step for calling an indirect function is to check if the index is within range by calling \texttt{\_\_sable\_table\_guard} built-in function. If the index is within range, we then compare the expecting function type with the actual indirect function type with \texttt{\_\_sable\_table\_check}. Note that this built-in function also checks if the entry is a null function. If so, it will report an exception. The SableWasm backend code generator uses a similar technique to encode the expected function type into a null-terminated string, as we have seen in section 5.1. After we make sure the indirect function is valid, we can now fetch the context pointer and function address pointer by using two getter functions, \texttt{\_\_sable\_table\_context} and \texttt{\_\_sable\_table\_function}. Before we invoke the function, we need to check if the function is a host function. A quick reminder, SableWasm will set context pointers for all host functions as null pointers, and when invoking a host function, we need to pass the current instance object pointer as the context pointer. The SableWasm code generator choose the correct context pointer by using a pair of \texttt{icmp} and \texttt{select} instruction. After selecting the correct context pointer, the indirect function is straightforward by casting the function code address into the function pointer and invoking it appropriately. One may notice that the indirect function call in SableWasm is costly and involves multiple function calls. WebAssembly specification does not require indirect function call efficiency, and later in our benchmark, we determine that indirect function calls are not a performance bottleneck. Hence, the SableWasm code generator focus on extensibility rather than performance.

\paragraph{SIMD operation}
The last translation pattern we will cover in the section is the SIMD operations. For most of the SIMD operations, the SableWasm backend code generator maps to their LLVM counterparts. However, one challenge arises when translating SableWasm MIR into LLVM intermediate representation around the type system. In section 4.3.3, we presented the type system for SableWasm MIR. A quick reminder, the SableWasm MIR follows WebAssembly's design by erasing the shape information from the vector values and depending on instructions to interpret them correctly. However, LLVM intermediate representation does require shape information for vectors. Hence, when lowering SableWasm MIR into LLVM intermediate representation, the SableWasm backend code generator needs to insert cast instructions when required. For most of the numerical instructions, this is pretty trivial. The backend code generator will first infer an LLVM vector type based on the SableWasm instruction shape information. For example, \texttt{v128.add i16x4} implies that the operand must have type \texttt{<4 x i16>} in LLVM. In the case where the shape type is unsuitable, the SableWasm backend code generator will insert a bit cast, \texttt{bitcast to}. The bit cast operation is always valid as, in the current version of SableWasm MIR, we only work with 128-bit vectors. However, there are still several corner cases in this strategy. What type should we assign to Phi nodes when merging vectors from multiple control-flow? Also, what type should we assign for load instruction when shape information is still not yet available? The SableWasm backend code generator takes advantage of the fact that integer types in LLVM can be arbitrarily long, and more specifically, 128-bit integer, \texttt{i128}, is a valid type in LLVM. The SableWasm backend code generator will always use \texttt{i128} as a default type in these corner cases. For example, for load instruction for SableWasm vectors, the code generator will emit a \texttt{load} instruction with \texttt{i128} type, and later when any instruction takes the value as the operand, it will setup the bit cast instruction accordingly.