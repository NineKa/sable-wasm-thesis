\begin{spacing}{1.4}
    \textit{WebAssembly} est un langage relativement nouveau, introduit pour améliorer les performances des charges de travail gourmandes en calcul dans les applications Web. Il offre un bytecode binaire compact destiné à permettre une compilation rapide et des opportunités d'optimisation améliorées sur des langages Web dynamiques comme JavaScript. Ces propriétés, cependant, en font également une cible intéressante pour l'exécution statique, permettant au code Web de s'exécuter en dehors d'un navigateur ainsi que dans celui-ci. Dans cette thèse, nous décrivons \textit{SableWasm}, un système de compilation statique à passes multiples qui traduit les applications WebAssembly en bac à sable en bibliothèques partagées natives. Notre travail couvre plusieurs aspects différents de la conception du compilateur. Premièrement, nous fournissons un cadre d'analyse et de validation de module WebAssembly efficace et extensible, avec une vitesse d'exécution et une empreinte mémoire améliorées par rapport à la ligne de base de référence. Nous définissons ensuite une représentation intermédiaire de niveau intermédiaire et construisons un cadre d'analyse et de transformation. Nous explorons plusieurs analyses classiques de flux de données, telles que la construction de l'arbre dominateur et la numérotation des valeurs locales dans le cadre, et identifions en outre plusieurs opportunités d'optimisation spécifiques à WebAssembly, que nous abordons via des passes de transformation personnalisées, telles que l'élimination des variables locales redondantes. SableWasm intègre également plusieurs propositions d'extension en cours, y compris l'extension d'opération de vecteur SIMD. Le code intermédiaire optimisé est ensuite converti en code natif via une implémentation backend à l'aide du framework de compilateur LLVM et d'un runtime qui permet aux programmes C/C++ d'interagir directement avec le module WebAssembly. Enfin, nous évaluons SableWasm en comparant plusieurs suites de tests bien connues et observons une amélioration des performances par rapport à l'implémentation de base.
\end{spacing}